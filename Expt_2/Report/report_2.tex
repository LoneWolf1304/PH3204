\documentclass{scrartcl}
\usepackage{pgfplots}
\usepackage{Style_File}
\usepackage{circuitikz}
\usepackage{fancyhdr}
\usepackage{array}
\newcolumntype{P}[1]{>{\centering\arraybackslash}p{#1}}
% Recommended preamble:
\usetikzlibrary{arrows.meta}
\usetikzlibrary{backgrounds}
\usepgfplotslibrary{patchplots}
\usepgfplotslibrary{fillbetween}
\pgfplotsset{%
    layers/standard/.define layer set={%
        background,axis background,axis grid,axis ticks,axis lines,axis tick labels,pre main,main,axis descriptions,axis foreground%
    }{
        grid style={/pgfplots/on layer=axis grid},%
        tick style={/pgfplots/on layer=axis ticks},%
        axis line style={/pgfplots/on layer=axis lines},%
        label style={/pgfplots/on layer=axis descriptions},%
        legend style={/pgfplots/on layer=axis descriptions},%
        title style={/pgfplots/on layer=axis descriptions},%
        colorbar style={/pgfplots/on layer=axis descriptions},%
        ticklabel style={/pgfplots/on layer=axis tick labels},%
        axis background@ style={/pgfplots/on layer=axis background},%
        3d box foreground style={/pgfplots/on layer=axis foreground},%
    },
}


\setlength{\headheight}{0.75in}
\setlength{\oddsidemargin}{0in}
        \setlength{\evensidemargin}{0in}
        \setlength{\textwidth}{6.5in}
        \setlength{\headwidth}{7.3in}
        \setlength{\textheight}{8.75in}
        \rfoot{\thepage}
        \renewcommand{\headrulewidth}{0pt} % Remove the header line
        \renewcommand{\footrulewidth}{0pt}
\fancyhead[L,C]{}
\fancyhead[L]{PH3204: Experiment 2}
\fancyhead[R]{\thepage}
\fancyhead[C]{Transistor Characteristics}
\fancyfoot[C]{\thepage}
\fancyfoot[R,L]{}
\pagestyle{fancy}
\renewcommand{\headrulewidth}{0.4pt}


\usepackage{siunitx}
\usepackage{longtable} 
\usepackage[left = 0.7in,
right = 0.7in,
bottom = 0.9in,
top = 0.9in,
a4paper]{geometry}

\title{
        \Large\textsc{Experiment 02: }
        \huge\textbf{Study of Transistor Characteristics} \\
}


\author{{\Large Sagnik Seth} -\   \texttt{22MS026}\\ ({\small Subgroup - A7}) }
\date{}


\begin{document}
\maketitle
\section{Aim}
To obtain the input and output characteristics of a transistor in CE configuration.
% \section{Theory}
% HI
 \section{Theory}

\subsection{Bipolar Junction Transistor (BJT)}
The Bipolar Junction Transistor (BJT) is essentially a combination of two PN junction diodes.  According to the order in which the diodes are connected, we have npn and pnp transistors. The emitter region is much smaller than the collector and much more strongly doped while the base is very thing and very lightly doped. A large collector is needed since it should have more surface area for dissipating the power generated there.\\[0.3cm]
\begin{figure}[H]
    \centering


    \tikzset{every picture/.style={line width=0.75pt}} %set default line width to 0.75pt        

    \begin{tikzpicture}[x=0.75pt,y=0.75pt,yscale=-1,xscale=1]
    %uncomment if require: \path (0,226); %set diagram left start at 0, and has height of 226
    
    %Shape: Rectangle [id:dp45633474700626153] 
    \draw  [fill={rgb, 255:red, 255; green, 255; blue, 255 }  ,fill opacity=1 ] (95.6,52) -- (168.81,52) -- (168.81,96.74) -- (95.6,96.74) -- cycle ;
    %Shape: Rectangle [id:dp14383516673441366] 
    \draw  [fill={rgb, 255:red, 255; green, 255; blue, 255 }  ,fill opacity=1 ] (195.79,52) -- (269,52) -- (269,96.74) -- (195.79,96.74) -- cycle ;
    %Shape: Rectangle [id:dp555436172774842] 
    \draw  [fill={rgb, 255:red, 155; green, 155; blue, 155 }  ,fill opacity=0.67 ] (168.81,52) -- (196.26,52) -- (196.26,96.74) -- (168.81,96.74) -- cycle ;
    %Shape: Rectangle [id:dp25579750012111335] 
    \draw  [fill={rgb, 255:red, 155; green, 155; blue, 155 }  ,fill opacity=0.67 ] (385.38,52) -- (458.59,52) -- (458.59,96.74) -- (385.38,96.74) -- cycle ;
    %Shape: Rectangle [id:dp7671828884714951] 
    \draw  [fill={rgb, 255:red, 155; green, 155; blue, 155 }  ,fill opacity=0.67 ] (486.04,52) -- (559.25,52) -- (559.25,96.74) -- (486.04,96.74) -- cycle ;
    %Shape: Rectangle [id:dp5338560444707424] 
    \draw  [fill={rgb, 255:red, 255; green, 255; blue, 255 }  ,fill opacity=1 ] (458.59,52) -- (486.04,52) -- (486.04,96.74) -- (458.59,96.74) -- cycle ;
    %Straight Lines [id:da24796501054461773] 
    \draw    (170.7,159.64) -- (202.37,159.64) ;
    %Straight Lines [id:da6481562502442813] 
    \draw    (170.14,155.72) -- (160.15,121.4) ;
    \draw [shift={(170.7,157.64)}, rotate = 253.76] [color={rgb, 255:red, 0; green, 0; blue, 0 }  ][line width=0.75]    (10.93,-3.29) .. controls (6.95,-1.4) and (3.31,-0.3) .. (0,0) .. controls (3.31,0.3) and (6.95,1.4) .. (10.93,3.29)   ;
    %Straight Lines [id:da7575975492052729] 
    \draw    (202.37,159.64) -- (213.98,123.4) ;
    %Straight Lines [id:da35919367940892855] 
    \draw    (451.35,159.64) -- (480.25,159.64) ;
    %Straight Lines [id:da5671054714723539] 
    \draw    (451.35,159.64) -- (442.24,126.45) ;
    \draw [shift={(441.71,124.52)}, rotate = 74.66] [color={rgb, 255:red, 0; green, 0; blue, 0 }  ][line width=0.75]    (10.93,-4.9) .. controls (6.95,-2.3) and (3.31,-0.67) .. (0,0) .. controls (3.31,0.67) and (6.95,2.3) .. (10.93,4.9)   ;
    %Straight Lines [id:da2022169589051811] 
    \draw    (480.25,159.64) -- (490.85,124.52) ;
    %Straight Lines [id:da04159202121636396] 
    \draw    (186.53,159.64) -- (186.5,182.37) ;
    %Straight Lines [id:da31155623971098534] 
    \draw    (465.8,159.64) -- (465.32,181.06) ;
    
    % Text Node
    \draw (418.48,65.31) node [anchor=north west][inner sep=0.75pt]   [align=left] {\textbf{n}};
    % Text Node
    \draw (515.86,65.37) node [anchor=north west][inner sep=0.75pt]   [align=left] {\textbf{n}};
    % Text Node
    \draw (179.15,65.31) node [anchor=north west][inner sep=0.75pt]  [color={rgb, 255:red, 0; green, 0; blue, 0 }  ,opacity=1 ] [align=left] {\textbf{n}};
    % Text Node
    \draw (223.73,65.31) node [anchor=north west][inner sep=0.75pt]   [align=left] {\textcolor[rgb]{0.96,0.03,0.03}{\textbf{p}}};
    % Text Node
    \draw (127.53,65.31) node [anchor=north west][inner sep=0.75pt]   [align=left] {\textcolor[rgb]{0.96,0.03,0.03}{\textbf{p}}};
    % Text Node
    \draw (467.75,65.31) node [anchor=north west][inner sep=0.75pt]   [align=left] {\textcolor[rgb]{0.96,0.03,0.03}{\textbf{p}}};
    % Text Node
    \draw (140.52,113.41) node [anchor=north west][inner sep=0.75pt]   [align=left] {E};
    % Text Node
    \draw (221.55,114.53) node [anchor=north west][inner sep=0.75pt]   [align=left] {C};
    % Text Node
    \draw (180.41,187.23) node [anchor=north west][inner sep=0.75pt]   [align=left] {B};
    % Text Node
    \draw (419.74,111.17) node [anchor=north west][inner sep=0.75pt]   [align=left] {E};
    % Text Node
    \draw (459.63,186.12) node [anchor=north west][inner sep=0.75pt]   [align=left] {B};
    % Text Node
    \draw (497.26,113.41) node [anchor=north west][inner sep=0.75pt]   [align=left] {C};
    
    
    \end{tikzpicture}
    

\end{figure}

\noindent
It is a three terminal device with the terminals named as emitter, base and collector but to act as an amplifier, we need four terminals. Hence, we make one terminal common. The most commonly studied configuration is the common emitter configuration where the emmiter is common to both input and output.\\[0.3cm]
A BJT works in three modes: Active, Saturation and Cutoff Modes, according to which junction is forward or reverse biased.  
\begin{itemize}
    \item \textbf{Active Mode:} BE junction is forward biased and BC junction is reverse biased. In this region, transistor acts as an amplifier.
    \item \textbf{Saturation Mode:} BE junction is forward biased and BC junction is forward biased. 
    \item \textbf{Cutoff Mode:} BE junction is reverse biased and BC junction is reverse biased. 
\end{itemize}
\subsection{Common Emitter Configuration of BJT}

\begin{figure}[H]    
    \centering
    \begin{circuitikz}[american voltages ]
        \draw
        % Transistor
        (0,0) node[npn, anchor=E, tr circle, fill = cyan!40!white] (Q) {}
        (Q.B) node[above left] {B}
        (Q.C) node[right] {C}
        (Q.E) node[below left] {E}
        % Base resistor
        (Q.B) -- (-2,0.75) to[R, l=$\mathrm{R_B}$] (-4,0.75) to [rmeter, t = $\mathrm{\mu A}$,fill = yellow!60!white]  (-6,0.75)
        to[battery, v_=$\mathrm{V_{BB}}$] (-6,-2) -- (0,-2)
    
        % Collector resistor
        (Q.C) -- (0,2) to[R, l=$\mathrm{R_C}$] (4,2)
        to [rmeter, t= $\mathrm{mA}$, fill = yellow!60!white] (5,2) -- (7,2)
        to [battery, v=$\mathrm{V_{CC}}$] (7,-2) -- (0,-2) 
    
        % Ground connection
        (Q.E) -- (0,-2) node[ground] {};
    \end{circuitikz}   
    \caption{Circuit diagram of a common emitter configuration of a NPN transistor}  
\end{figure}


The above figure shows the common emitter configuration of the transistor. From the figure, we can see: 
$$\mathrm{I_E = I_B+ I_C}$$
Normally when used as an amplifier,
the base emitter junction is kept in forward bias and the collector base junction is kept
in reverse bias. So, due to the forward bias of the base emitter junction electrons go
from emitter to the base, this electrons get affected by the reverse bias of the collector
base junction and as the base is very thin and lightly doped most of the electrons go
to the collector. So the emitter current is almost equal to the collector current. We define two parameters $\beta$ and $\alpha$ for the transistor as follows:
$$\beta = \mathrm{\frac{I_C}{I_B}}$$
$$\alpha = \mathrm{\frac{I_C}{I_E}}$$
From this, we get a relation between $\alpha$ and $\beta$ as:
$$\alpha = \mathrm{\frac{\beta}{1+\beta}} \implies \beta = \frac{\alpha}{1-\alpha} $$
Generally $\beta$ is much greater than 1 while $\alpha$ is very close to but less than 1 .\\[0.3cm]
In the CE configuration the input current and voltage are $\mathrm{I_B}$ and $\mathrm{V_{BE}}$, while the output current and voltage are $\mathrm{I_C}$ and $\mathrm{V_{CE}}$. Thus, for the input characteristics, we plot $\mathrm{I_B}$ vs $\mathrm{V_{BE}}$ and for the output characteristics, we plot $\mathrm{I_C}$ vs $\mathrm{V_{CE}}$.\\[0.2cm]
\section{Data and Calculations}
\subsection{Input Characteristics}
\begin{table}[H]
    \centering
    \begin{tabular}{|c|c|c|c|c|c|}
        \hline
    \textbf{V\textsubscript{BB} (V)}  & \textbf{V\textsubscript{BE} (V)} &\textbf{I\textsubscript{B} (mA)} & \textbf{I\textsubscript{C} (mA)} & \textbf{V\textsubscript{CC} (V)} &  \textbf{V\textsubscript{CE} (V)} \\
    \hline 
    0         & 0.036     & 0            & 0            & 2     & 2             \\ \hline
    0.1       & 0.178     & 0            & 0            & 2     & 2             \\ \hline
    0.2       & 0.251     & 0            & 0            & 2     & 2             \\ \hline
    0.3       & 0.378     & 0            & 0            & 2     & 2             \\ \hline
    0.4       & 0.449     & 0            & 0            & 2     & 2             \\ \hline
    0.5       & 0.589     & 3            & 5            & 2     & 2             \\ \hline
    0.6       & 0.655     & 28           & 4.7          & 2     & 2             \\ \hline 
    0.7       & 0.698     & 91           & 16.3         & 2     & 2             \\ \hline
    0.8       & 0.729     & 182          & 32.6         & 2     & 2             \\ \hline
    0.9       & 0.738     & 226          & 41.1         & 2     & 2             \\ \hline
    1         & 0.762     & 338          & 61.6         & 2     & 2             \\ \hline 
    1.1       & 0.778     & 414          & 76.1         & 2     & 2             \\ \hline
    1.2       & 0.792     & 499          & 92.3         & 2     & 2             \\ 
    \hline
    \end{tabular}
    \end{table}

    \begin{table}[H]
        \centering
        \begin{tabular}{|c|c|c|c|c|c|}
        \hline
        \textbf{V\textsubscript{BB} (V)}  & \textbf{V\textsubscript{BE} (V)} &\textbf{I\textsubscript{B} (mA)} & \textbf{I\textsubscript{C} (mA)} & \textbf{V\textsubscript{CC} (V)} &  \textbf{V\textsubscript{CE} (V)}  \\
    \hline
        0     & 0.038 & 0            & 0            & 3     & 3             \\ \hline
        0.1   & 0.19  & 0            & 0            & 3     & 3             \\ \hline
        0.2   & 0.259 & 0            & 0            & 3     & 3             \\ \hline
        0.3   & 0.355 & 1            & 0            & 3     & 3             \\ \hline
        0.4   & 0.466 & 1            & 0            & 3     & 3             \\ \hline
        0.5   & 0.607 & 9            & 1.6          & 3     & 3             \\ \hline
        0.6   & 0.642 & 27           & 5            & 3     & 3             \\ \hline
        0.7   & 0.684 & 88           & 16.2         & 3     & 3             \\ \hline
        0.8   & 0.71  & 158          & 29.4         & 3     & 3             \\ \hline
        0.9   & 0.729 & 242          & 45.9         & 3     & 3             \\ \hline
        1     & 0.749 & 334          & 64.8         & 3     & 3             \\ \hline
        1.1   & 0.767 & 449          & 87.2         & 3     & 3             \\ \hline
        1.2   & 0.765 & 468          & 93           & 3     & 3             \\
        \hline           
        \end{tabular}
        \end{table}


% Recommended preamble:
% \usetikzlibrary{arrows.meta}
% \usetikzlibrary{backgrounds}
% \usepgfplotslibrary{patchplots}
% \usepgfplotslibrary{fillbetween}
% \pgfplotsset{%
%     layers/standard/.define layer set={%
%         background,axis background,axis grid,axis ticks,axis lines,axis tick labels,pre main,main,axis descriptions,axis foreground%
%     }{
%         grid style={/pgfplots/on layer=axis grid},%
%         tick style={/pgfplots/on layer=axis ticks},%
%         axis line style={/pgfplots/on layer=axis lines},%
%         label style={/pgfplots/on layer=axis descriptions},%
%         legend style={/pgfplots/on layer=axis descriptions},%
%         title style={/pgfplots/on layer=axis descriptions},%
%         colorbar style={/pgfplots/on layer=axis descriptions},%
%         ticklabel style={/pgfplots/on layer=axis tick labels},%
%         axis background@ style={/pgfplots/on layer=axis background},%
%         3d box foreground style={/pgfplots/on layer=axis foreground},%
%     },
% }
\begin{figure}[H]
    \centering
\begin{tikzpicture}[/tikz/background rectangle/.style={fill={rgb,1:red,1.0;green,1.0;blue,1.0}, fill opacity={1.0}, draw opacity={1.0}}, show background rectangle, scale=0.7]
\begin{axis}[point meta max={nan}, point meta min={nan}, legend cell align={left}, legend columns={1}, title={}, title style={at={{(0.5,1)}}, anchor={south}, font={{\fontsize{14 pt}{18.2 pt}\selectfont}}, color={rgb,1:red,0.0;green,0.0;blue,0.0}, draw opacity={1.0}, rotate={0.0}, align={center}}, legend style={color={rgb,1:red,0.0;green,0.0;blue,0.0}, draw opacity={1.0}, line width={1}, solid, fill={rgb,1:red,1.0;green,1.0;blue,1.0}, fill opacity={1.0}, text opacity={1.0}, font={{\fontsize{8 pt}{10.4 pt}\selectfont}}, text={rgb,1:red,0.0;green,0.0;blue,0.0}, cells={anchor={center}}, at={(1.02, 1)}, anchor={north west}}, axis background/.style={fill={rgb,1:red,1.0;green,1.0;blue,1.0}, opacity={1.0}}, anchor={north west}, xshift={1.0mm}, yshift={-1.0mm}, width={145.4mm}, height={99.6mm}, scaled x ticks={false}, xlabel={$\mathrm{V_{BE} \ (V)}$}, x tick style={color={rgb,1:red,0.0;green,0.0;blue,0.0}, opacity={1.0}}, x tick label style={color={rgb,1:red,0.0;green,0.0;blue,0.0}, opacity={1.0}, rotate={0}}, xlabel style={at={(ticklabel cs:0.5)}, anchor=near ticklabel, at={{(ticklabel cs:0.5)}}, anchor={near ticklabel}, font={{\fontsize{11 pt}{14.3 pt}\selectfont}}, color={rgb,1:red,0.0;green,0.0;blue,0.0}, draw opacity={1.0}, rotate={0.0}}, xmajorgrids={true}, xmin={0.013319999999999999}, xmax={0.8146800000000001}, xticklabels={{$0.1$,$0.2$,$0.3$,$0.4$,$0.5$,$0.6$,$0.7$,$0.8$}}, xtick={{0.1,0.2,0.3,0.4,0.5,0.6,0.7,0.8}}, xtick align={inside}, xticklabel style={font={{\fontsize{8 pt}{10.4 pt}\selectfont}}, color={rgb,1:red,0.0;green,0.0;blue,0.0}, draw opacity={1.0}, rotate={0.0}}, x grid style={color={rgb,1:red,0.0;green,0.0;blue,0.0}, draw opacity={0.1}, line width={3}, solid}, axis x line*={left}, x axis line style={color={rgb,1:red,0.0;green,0.0;blue,0.0}, draw opacity={1.0}, line width={1}, solid}, scaled y ticks={false}, ylabel={$\mathrm{I_B} \ (\mu A)$}, y tick style={color={rgb,1:red,0.0;green,0.0;blue,0.0}, opacity={1.0}}, y tick label style={color={rgb,1:red,0.0;green,0.0;blue,0.0}, opacity={1.0}, rotate={0}}, ylabel style={at={(ticklabel cs:0.5)}, anchor=near ticklabel, at={{(ticklabel cs:0.5)}}, anchor={near ticklabel}, font={{\fontsize{11 pt}{14.3 pt}\selectfont}}, color={rgb,1:red,0.0;green,0.0;blue,0.0}, draw opacity={1.0}, rotate={0.0}}, ymajorgrids={true}, ymin={-14.970000000000027}, ymax={513.97}, yticklabels={{$0$,$50$,$100$,$150$,$200$,$250$,$300$,$350$,$400$,$450$,$500$}}, ytick={{0.0,50.0,100.0,150.0,200.0,250.0,300.0,350.0,400.0,450.0,500.0}}, ytick align={inside}, yticklabel style={font={{\fontsize{8 pt}{10.4 pt}\selectfont}}, color={rgb,1:red,0.0;green,0.0;blue,0.0}, draw opacity={1.0}, rotate={0.0}}, y grid style={color={rgb,1:red,0.0;green,0.0;blue,0.0}, draw opacity={0.1}, line width={3}, solid}, axis y line*={left}, y axis line style={color={rgb,1:red,0.0;green,0.0;blue,0.0}, draw opacity={1.0}, line width={1}, solid}, colorbar={false}]
    \addplot[color={rgb,1:red,1.0;green,0.0;blue,0.0}, name path={61}, draw opacity={1.0}, line width={2}, solid, mark={*}, mark size={2.25 pt}, mark repeat={1}, mark options={color={rgb,1:red,0.0;green,0.0;blue,0.0}, draw opacity={1.0}, fill={rgb,1:red,1.0;green,0.0;blue,0.0}, fill opacity={1.0}, line width={0.375}, rotate={0}, solid}]
        table[row sep={\\}]
        {
            \\
            0.036  0.0  \\
            0.178  0.0  \\
            0.251  0.0  \\
            0.378  0.0  \\
            0.449  0.0  \\
            0.589  3.0  \\
            0.655  28.0  \\
            0.698  91.0  \\
            0.729  182.0  \\
            0.738  226.0  \\
            0.762  338.0  \\
            0.778  414.0  \\
            0.792  499.0  \\
        }
        ;
    \addlegendentry {$\mathrm{VCE=2 \ V}$}
    \addplot[color={rgb,1:red,0.0;green,0.0;blue,0.0}, name path={62}, draw opacity={1.0}, line width={2}, solid, mark={*}, mark size={2.25 pt}, mark repeat={1}, mark options={color={rgb,1:red,0.0;green,0.0;blue,0.0}, draw opacity={1.0}, fill={rgb,1:red,0.0;green,0.0;blue,0.0}, fill opacity={1.0}, line width={0.375}, rotate={0}, solid}]
        table[row sep={\\}]
        {
            \\
            0.038  0.0  \\
            0.19  0.0  \\
            0.259  0.0  \\
            0.355  1.0  \\
            0.466  1.0  \\
            0.607  9.0  \\
            0.642  27.0  \\
            0.684  88.0  \\
            0.71  158.0  \\
            0.729  242.0  \\
            0.749  334.0  \\
            0.767  449.0  \\
            0.765  468.0  \\
        }
        ;
    \addlegendentry {$\mathrm{VCE=3 \ V}$}
\end{axis}
\end{tikzpicture}
\caption{Input characteristics of a BJT transistor}
\end{figure}
\subsection{Output Characteristics}

\begin{table}[H]
        \centering
        \begin{tabular}{|c|c|c|c|c|}
        \hline
        \textbf{V\textsubscript{BB} (V)} & 
        \textbf{I\textsubscript{B} (\textmu A)} & 
        \textbf{V\textsubscript{CC} (V)} & 
        \textbf{V\textsubscript{CE} (mV)} & 
        \textbf{I\textsubscript{C} (\textmu A)} \\ \hline
            0.5 & 25 & 0.0 & 0.004 & 10 \\ \hline
            0.5 & 25 & 0.5 & 0.04 & 506 \\ \hline
            0.5 & 25 & 1.0 & 0.057 & 922 \\ \hline
            0.5 & 25 & 1.5 & 0.069 & 1308 \\ \hline
            0.5 & 25 & 2.0 & 0.081 & 1784 \\ \hline
            0.5 & 25 & 2.5 & 0.099 & 2390 \\ \hline
            0.5 & 25 & 3.0 & 0.107 & 2880 \\ \hline
            0.5 & 25 & 3.5 & 0.127 & 3350 \\ \hline
            0.5 & 25 & 4.0 & 0.15 & 3780 \\ \hline
            0.6 & 25 & 4.2 & 0.172 & 3960 \\ \hline
            0.6 & 25 & 4.4 & 0.2 & 4150 \\ \hline
            0.6 & 25 & 4.6 & 0.228 & 4340 \\ \hline
            0.6 & 25 & 4.8 & 0.42 & 4350 \\ \hline
            0.6 & 25 & 4.7 & 0.349 & 4380 \\ \hline
            0.6 & 25 & 5.0 & 0.49 & 4340 \\ \hline
            0.6 & 25 & 5.2 & 0.696 & 4460 \\ \hline
            0.6 & 25 & 5.5 & 0.978 & 4460 \\ \hline
        \end{tabular}
        \caption{Data obtained for output characteristics at I\textsubscript{B} = 25 \textmu A }
    \end{table}


    \begin{table}[H]
        \centering
        \begin{tabular}{|c|c|c|c|c|}
        \hline
        \textbf{V\textsubscript{CC} (V)} & 
        \textbf{V\textsubscript{BB} (V)} & 
        \textbf{I\textsubscript{B} (\textmu A)} &
        
        \textbf{V\textsubscript{CE} (mV)} & 
        \textbf{I\textsubscript{C} (\textmu A)} \\ \hline
            0 & 0.5 & 30 & 0.007 & 37 \\ \hline
            0.1 & 0.5 & 30 & 0.02 & 166 \\ \hline
            0.2 & 0.5 & 30 & 0.022 & 214 \\ \hline
            0.3 & 0.5 & 30 & 0.028 & 331 \\ \hline
            0.4 & 0.5 & 30 & 0.03 & 362 \\ \hline
            0.5 & 0.5 & 30 & 0.037 & 513 \\ \hline
            0.6 & 0.5 & 30 & 0.04 & 610 \\ \hline
            0.7 & 0.5 & 30 & 0.043 & 671 \\ \hline
            0.8 & 0.5 & 30 & 0.045 & 737 \\ \hline
            0.9 & 0.5 & 30 & 0.049 & 829 \\ \hline
            1 & 0.5 & 30 & 0.05 & 884 \\ \hline
            1.1 & 0.5 & 30 & 0.054 & 1016 \\ \hline
            1.2 & 0.5 & 30 & 0.058 & 1118 \\ \hline
            1.3 & 0.5 & 30 & 0.062 & 1294 \\ \hline
            1.4 & 0.5 & 30 & 0.063 & 1296 \\ \hline
            1.5 & 0.5 & 30 & 0.064 & 1335 \\ \hline
            1.8 & 0.5 & 30 & 0.072 & 1599 \\ \hline
            2 & 0.5 & 30 & 0.073 & 1770 \\ \hline
            3 & 0.5 & 30 & 0.098 & 2950 \\ \hline
            3.5 & 0.6 & 30 & 0.11 & 3420 \\ \hline
            4 & 0.6 & 30 & 0.122 & 3850 \\ \hline
            4.5 & 0.6 & 30 & 0.139 & 4350 \\ \hline
            4.6 & 0.6 & 30 & 0.14 & 4430 \\ \hline
            4.8 & 0.6 & 30 & 0.156 & 4630 \\ \hline
            4.9 & 0.6 & 30 & 0.166 & 4690 \\ \hline
            5 & 0.6 & 30 & 0.198 & 4820 \\ \hline
            5.5 & 0.6 & 30 & 0.27 & 5070 \\ \hline
            5.6 & 0.6 & 30 & 0.42 & 5110 \\ \hline
            5.8 & 0.6 & 30 & 0.58 & 5130 \\ \hline
            6 & 0.6 & 30 & 0.89 & 5090 \\ \hline
        \end{tabular}
        \caption{Data obtained for output characteristics at I\textsubscript{B} = 30 \textmu A }
    \end{table}
    \begin{table}[H]
        \centering
        \begin{tabular}{|c|c|c|c|c|}
        \hline
        \textbf{V\textsubscript{BB} (V)} & 
        \textbf{I\textsubscript{B} (\textmu A)} & 
        \textbf{V\textsubscript{CC} (V)} & 
        \textbf{V\textsubscript{CE} (mV)} & 
        \textbf{I\textsubscript{C} (\textmu A)} \\ \hline
            0.5 & 40 & 0.0 & 0.004 & 12 \\ \hline
            0.5 & 40 & 1.0 & 0.04 & 866 \\ \hline
            0.5 & 40 & 2.0 & 0.061 & 1724 \\ \hline
            0.5 & 40 & 3.0 & 0.08 & 2890 \\ \hline
            0.5 & 40 & 4.0 & 0.094 & 3740 \\ \hline
            0.6 & 40 & 5.0 & 0.109 & 4730 \\ \hline
            0.6 & 40 & 6.0 & 0.134 & 5730 \\ \hline
            0.6 & 40 & 6.5 & 0.147 & 6180 \\ \hline
            0.6 & 40 & 7.0 & 0.167 & 6700 \\ \hline
            0.6 & 40 & 7.2 & 0.188 & 6850 \\ \hline
            0.6 & 40 & 7.4 & 0.388 & 6890 \\ \hline
            0.6 & 40 & 7.6 & 0.533 & 6900 \\ \hline
            0.6 & 40 & 8.0 & 0.921 & 7000 \\ \hline
        \end{tabular}
        \caption{Data obtained for output characteristics at I\textsubscript{B} = 40 \textmu A }
    \end{table}
    \begin{figure}[H]
    \centering
\begin{tikzpicture}[/tikz/background rectangle/.style={fill={rgb,1:red,1.0;green,1.0;blue,1.0}, fill opacity={1.0}, draw opacity={1.0}}, show background rectangle, scale=0.8]
\begin{axis}[point meta max={nan}, point meta min={nan}, legend cell align={left}, legend columns={1}, title={Output Characteristics of Transistor}, title style={at={{(0.5,1)}}, anchor={south}, font={{\fontsize{14 pt}{18.2 pt}\selectfont}}, color={rgb,1:red,0.0;green,0.0;blue,0.0}, draw opacity={1.0}, rotate={0.0}, align={center}}, legend style={color={rgb,1:red,0.0;green,0.0;blue,0.0}, draw opacity={1.0}, line width={1}, solid, fill={rgb,1:red,1.0;green,1.0;blue,1.0}, fill opacity={1.0}, text opacity={1.0}, font={{\fontsize{8 pt}{10.4 pt}\selectfont}}, text={rgb,1:red,0.0;green,0.0;blue,0.0}, cells={anchor={center}}, at={(1.02, 1)}, anchor={north west}}, axis background/.style={fill={rgb,1:red,1.0;green,1.0;blue,1.0}, opacity={1.0}}, anchor={north west}, xshift={1.0mm}, yshift={-1.0mm}, width={145.4mm}, height={99.6mm}, scaled x ticks={false}, xlabel={$\mathrm{V_{CE}}\  \mathrm{(V)}$}, x tick style={color={rgb,1:red,0.0;green,0.0;blue,0.0}, opacity={1.0}}, x tick label style={color={rgb,1:red,0.0;green,0.0;blue,0.0}, opacity={1.0}, rotate={0}}, xlabel style={at={(ticklabel cs:0.5)}, anchor=near ticklabel, at={{(ticklabel cs:0.5)}}, anchor={near ticklabel}, font={{\fontsize{11 pt}{14.3 pt}\selectfont}}, color={rgb,1:red,0.0;green,0.0;blue,0.0}, draw opacity={1.0}, rotate={0.0}}, xmajorgrids={true}, xmin={-0.02522000000000002}, xmax={1.00722}, xticklabels={{$0.0$,$0.1$,$0.2$,$0.3$,$0.4$,$0.5$,$0.6$,$0.7$,$0.8$,$0.9$,$1.0$}}, xtick={{0.0,0.1,0.2,0.3,0.4,0.5,0.6,0.7,0.8,0.9,1.0}}, xtick align={inside}, xticklabel style={font={{\fontsize{8 pt}{10.4 pt}\selectfont}}, color={rgb,1:red,0.0;green,0.0;blue,0.0}, draw opacity={1.0}, rotate={0.0}}, x grid style={color={rgb,1:red,0.0;green,0.0;blue,0.0}, draw opacity={0.1}, line width={3}, solid}, axis x line*={left}, x axis line style={color={rgb,1:red,0.0;green,0.0;blue,0.0}, draw opacity={1.0}, line width={1}, solid}, scaled y ticks={false}, ylabel={$\mathrm{I_C\  (mA) }$}, y tick style={color={rgb,1:red,0.0;green,0.0;blue,0.0}, opacity={1.0}}, y tick label style={color={rgb,1:red,0.0;green,0.0;blue,0.0}, opacity={1.0}, rotate={0}}, ylabel style={at={(ticklabel cs:0.5)}, anchor=near ticklabel, at={{(ticklabel cs:0.5)}}, anchor={near ticklabel}, font={{\fontsize{11 pt}{14.3 pt}\selectfont}}, color={rgb,1:red,0.0;green,0.0;blue,0.0}, draw opacity={1.0}, rotate={0.0}}, ymajorgrids={true}, ymin={-199.70000000000027}, ymax={7209.700000000001}, yticklabels={{$0$,$1000$,$2000$,$3000$,$4000$,$5000$,$6000$}}, ytick={{0.0,1000.0,2000.0,3000.0,4000.0,5000.0,6000.0}}, ytick align={inside}, yticklabel style={font={{\fontsize{8 pt}{10.4 pt}\selectfont}}, color={rgb,1:red,0.0;green,0.0;blue,0.0}, draw opacity={1.0}, rotate={0.0}}, y grid style={color={rgb,1:red,0.0;green,0.0;blue,0.0}, draw opacity={0.1}, line width={3}, solid}, axis y line*={left}, y axis line style={color={rgb,1:red,0.0;green,0.0;blue,0.0}, draw opacity={1.0}, line width={1}, solid}, colorbar={false}]
    \addplot[color={rgb,1:red,1.0;green,0.0;blue,0.0}, name path={19}, draw opacity={1.0}, line width={2}, solid, mark={*}, mark size={2.25 pt}, mark repeat={1}, mark options={color={rgb,1:red,0.0;green,0.0;blue,0.0}, draw opacity={1.0}, fill={rgb,1:red,1.0;green,0.0;blue,0.0}, fill opacity={1.0}, line width={0.375}, rotate={0}, solid}]
        table[row sep={\\}]
        {
            \\
            0.004  10.0  \\
            0.04  506.0  \\
            0.057  922.0  \\
            0.069  1308.0  \\
            0.081  1784.0  \\
            0.099  2390.0  \\
            0.107  2880.0  \\
            0.127  3350.0  \\
            0.15  3780.0  \\
            0.172  3960.0  \\
            0.2  4150.0  \\
            0.228  4340.0  \\
            0.349  4380.0  \\
            0.42  4350.0  \\
            0.49  4340.0  \\
            0.696  4460.0  \\
            0.978  4460.0  \\
        }
        ;
    \addlegendentry {$\mathrm{I_B}=25\ \mathrm{\mu A}$}
    \addplot[color={rgb,1:red,0.0;green,0.0;blue,1.0}, name path={20}, draw opacity={1.0}, line width={2}, solid, mark={*}, mark size={2.25 pt}, mark repeat={1}, mark options={color={rgb,1:red,0.0;green,0.0;blue,0.0}, draw opacity={1.0}, fill={rgb,1:red,0.0;green,0.0;blue,1.0}, fill opacity={1.0}, line width={0.375}, rotate={0}, solid}]
        table[row sep={\\}]
        {
            \\
            0.007  37.0  \\
            0.02  166.0  \\
            0.022  214.0  \\
            0.028  331.0  \\
            0.03  362.0  \\
            0.037  513.0  \\
            0.04  610.0  \\
            0.043  671.0  \\
            0.045  737.0  \\
            0.049  829.0  \\
            0.05  884.0  \\
            0.054  1016.0  \\
            0.058  1118.0  \\
            0.062  1294.0  \\
            0.063  1296.0  \\
            0.064  1335.0  \\
            0.072  1599.0  \\
            0.073  1770.0  \\
            0.098  2950.0  \\
            0.11  3420.0  \\
            0.122  3850.0  \\
            0.139  4350.0  \\
            0.14  4430.0  \\
            0.156  4630.0  \\
            0.166  4690.0  \\
            0.198  4820.0  \\
            0.27  5070.0  \\
            0.42  5110.0  \\
            0.58  5130.0  \\
            0.89  5090.0  \\
        }
        ;
    \addlegendentry {$\mathrm{I_B}=30\ \mathrm{\mu A}$}
    \addplot[color={rgb,1:red,0.0;green,0.502;blue,0.0}, name path={21}, draw opacity={1.0}, line width={2}, solid, mark={*}, mark size={2.25 pt}, mark repeat={1}, mark options={color={rgb,1:red,0.0;green,0.0;blue,0.0}, draw opacity={1.0}, fill={rgb,1:red,0.0;green,0.502;blue,0.0}, fill opacity={1.0}, line width={0.375}, rotate={0}, solid}]
        table[row sep={\\}]
        {
            \\
            0.004  12.0  \\
            0.04  866.0  \\
            0.061  1724.0  \\
            0.08  2890.0  \\
            0.094  3740.0  \\
            0.109  4730.0  \\
            0.134  5730.0  \\
            0.147  6180.0  \\
            0.167  6700.0  \\
            0.188  6850.0  \\
            0.388  6890.0  \\
            0.533  6900.0  \\
            0.921  7000.0  \\
        }
        ;
    \addlegendentry {$\mathrm{I_B}=40\ \mathrm{\mu A}$}
\end{axis}
\end{tikzpicture}
\caption{Output characteristics of BJT}
\end{figure}

\section{Sources of Error}
\section{Discussion and Conclusion}

\end{document}