\documentclass{scrartcl}
\usepackage{pgfplots}
\usepackage{makecell}
\usepackage{multirow} % For merged cells
\usepackage{booktabs} % For better table formatting
\usepackage{calc}
\usepackage{Style_File}
\usepackage{circuitikz}
\usepackage{fancyhdr}
\usepackage{array}
\newcolumntype{P}[1]{>{\centering\arraybackslash}p{#1}}
% Recommended preamble:
\usetikzlibrary{arrows.meta}
\usetikzlibrary{backgrounds}
\usepgfplotslibrary{patchplots}
\usepgfplotslibrary{fillbetween}
\pgfplotsset{%
    layers/standard/.define layer set={%
        background,axis background,axis grid,axis ticks,axis lines,axis tick labels,pre main,main,axis descriptions,axis foreground%
    }{
        grid style={/pgfplots/on layer=axis grid},%
        tick style={/pgfplots/on layer=axis ticks},%
        axis line style={/pgfplots/on layer=axis lines},%
        label style={/pgfplots/on layer=axis descriptions},%
        legend style={/pgfplots/on layer=axis descriptions},%
        title style={/pgfplots/on layer=axis descriptions},%
        colorbar style={/pgfplots/on layer=axis descriptions},%
        ticklabel style={/pgfplots/on layer=axis tick labels},%
        axis background@ style={/pgfplots/on layer=axis background},%
        3d box foreground style={/pgfplots/on layer=axis foreground},%
    },
}


\setlength{\headheight}{0.75in}
\setlength{\oddsidemargin}{0in}
        \setlength{\evensidemargin}{0in}
        \setlength{\textwidth}{6.5in}
        \setlength{\headwidth}{7.3in}
        \setlength{\textheight}{8.75in}
        \rfoot{\thepage}
        \renewcommand{\headrulewidth}{0pt} % Remove the header line
        \renewcommand{\footrulewidth}{0pt}
\fancyhead[L,C]{}
\fancyhead[L]{PH3204: Experiment 5}
\fancyhead[R]{0\thepage}
\fancyhead[C]{ Boolean Algebra}
\fancyfoot[C]{0\thepage}
\fancyfoot[R,L]{}
\pagestyle{fancy}
\renewcommand{\headrulewidth}{0.4pt}


\usepackage{siunitx}
\usepackage{longtable} 
\usepackage[left = 0.7in,
right = 0.7in,
bottom = 0.9in,
top = 0.9in,
a4paper]{geometry}

\title{
        \Large\textsc{Experiment 05: }
        \huge\textbf{ Study of 555 Timer IC} \\
}


\author{{\Large Sagnik Seth} -\   \texttt{22MS026}\\ ({\small Subgroup - A7}) }
\date{}


\begin{document}
\maketitle
\section{Aim}
To  study the use of 555 Time IC as an astable
multivibrator.

\section{Theory}

\tikzstyle{icdev}=[draw, text width=5.5em, minimum height=7.3em, fill=cyan!40!white]
\begin{figure}[H]
    \centering
    \begin{tikzpicture}[every node/.style = {font = \tiny},american]
        \draw (0,4) node[left]{$\mathrm{V_{s}}$}  % from top Vcc to bottom Gnd
            to[short,o-] (0.8,4)
            to[/tikz/circuitikz/bipoles/length=0.7cm,R, l_=$\mathrm{R_1}$] (0.8,2.6) % set bipole device size
            to[/tikz/circuitikz/bipoles/length=0.7cm,R, l_=$\mathrm{R_2}$] (0.8,1.8)  to[/tikz/circuitikz/bipoles/length=0.7cm,C, l_=$\mathrm{C_1}$] (0.8,0.09) -- (0.8,0)
            to[short,-o] (0,0) node[left]{GND}
        ;
        \node (digichip) [icdev,xshift=3cm,yshift=2cm] {};
        \draw[blue] (3,2) node [align=center]{\large 555\\TIMER};  % position IC device body
    % top terminal lines/pins - 4 RESET, 8 Vcc
        \path [draw](0.8,4) -| (2.5,3.4) node[below]{RESET} node[above left]{4};
        \path [draw](2.5,4) -| (3.5,3.4) node[below]{$\mathrm{V_{s}}$} node[above left] {8};
    % bottom terminal lines/pins - 1 GND, 5 CTRL
        \path [draw](0.8,0) -| (2.5,0.6) node[above]{GND} node[below left]{1};
        \path [draw](2.5,0) -- (3.5,0)
            to[/tikz/circuitikz/bipoles/length=0.7cm,C](3.5,0.6)  
            node[above]{CTRL} node[below right]{\ \ 5}; % C = 10nf
    % leftside terminal lines/pins - 7 DIS, 6 THR, 2 TRG
        \draw (0.8,2.7) -- (1.83,2.7) node[right]{DIS} node[above left]{7}
             (1.83,2) node[right]{THR} node[above left]{6} |- (1.3,2) -- (1.3,1.5) --
            (0.8,1.5) -- (1.83,1.5) node[right]{TRG} node[above left]{2};
    % rightside terminal line/pin - 3 out
        \draw (4.17,2) node[left]{Out} -- (4.8,2) node[above left]{3} to[short,o-] (4.8,2);
    \end{tikzpicture}
    \caption{Circuit Diagram of 555 Timer IC}
\end{figure}

\section{Data and Calculation}
\section{Sources of Error}
\section{Discussion and Conclusion}
\end{document}
