\documentclass[12pt]{article}

\usepackage{amsmath}
\usepackage{graphicx}
\usepackage{multirow}
\usepackage{svg}
\usepackage{float}
\usepackage{longtable} 
\usepackage{siunitx}
\usepackage{circuitikz}
\usepackage[margin=1.2in]{geometry}


\begin{document}
	\title{Sub-Group: A-7 \\ Experiment 5: Study of 555 Timer IC}
	
	
	\author{Sayan Karmakar \\22MS163 }
	\date{}
	\maketitle

\section{Aim:}
Study of 555 Timer IC as Astable multivibrator with different frequency.
%=========================================================================================
%						   	        	THEORY											  
%=========================================================================================
\section{Theory:}
\textbf{Pin Functions: }
\begin{enumerate}
	\item \textbf{Trigger Input:} When it is less than $V_S/3$ it makes the output high $V_S$. It monitors the discharging of timing capacitor in astable circuit.
	\item \textbf{Threshold Input:} When it is greater than $2 V_S/ 3$, it makes the output low (0 V). But this only happens if the trigger input is more than $V_S/ 3$. If the trigger input is low then it forces the output to be high. This input monitors the charging of time capacitor in astable and monostable circuit.
	\item \textbf{Reset Input:} When it is less than 0.7 V, it makes the output low(0 V), overriding other inputs. When it is unnecessary it should be connected to the source voltage.
	\item \textbf{Control Input:} If there is a need to change the threshold voltage which is normally set to $2 V_S/3$, 	usually this is connected to 0 V with a very low capacitor of 0.01 $\mu$F to avoid electrical noise.
	\item \textbf{Discharge Pin: }
\end{enumerate}

\subsection{555/ 556 Astable Circuit:}
Time period ($T$) of the square wave is the time for one complete cycle. And frequency ($f$) of the wave is no. of complete cycles per second.

\begin{equation*}
	\begin{split}
		&T = 0.7 (R_1 + 2 R_2) C_1 \\
		&f = \frac{1.4}{(R_1 + 2 R_2) C_1}
	\end{split}
	\end{equation*}

Time period can be split into two part, when the output is high, \textbf{mark time} ($T_m$) and when the output is low,  \textbf{space time} ($T_s$).

\begin{equation*}
	\begin{split}
		&T_m = 0.7 (R_1 + R_2) C_1 \\
		&T_s = 0.7 R_2 C_1
	\end{split}.
\end{equation*}

%=========================================================================================
%									Data and Analysis									  
%=========================================================================================

\section{Data and Analysis}

\end{document}
