\documentclass[12pt]{article}

\usepackage{amsmath}
\usepackage{graphicx}
\usepackage{multirow}
\usepackage{svg}
\usepackage{float}
\usepackage{longtable}
\usepackage{booktabs}
\usepackage{siunitx}
\usepackage{circuitikz}
\usepackage[margin=1.2in]{geometry}


\begin{document}
	\title{Sub-Group: A-7 \\ Experiment 5: Study of 555 Timer IC}
	
	
	\author{Sayan Karmakar \\22MS163 }
	\date{}
	\maketitle

\section{Aim:}
Study of 555 Timer IC as Astable multivibrator with different frequency.
%=========================================================================================
%						   	        	THEORY											  
%=========================================================================================
\section{Theory:}
\textbf{Pin Functions: }
\begin{enumerate}
	\item \textbf{Trigger Input:} When it is less than $V_S/3$ it makes the output high $V_S$. It monitors the discharging of timing capacitor in astable circuit.
	\item \textbf{Threshold Input:} When it is greater than $2 V_S/ 3$, it makes the output low (0 V). But this only happens if the trigger input is more than $V_S/ 3$. If the trigger input is low then it forces the output to be high. This input monitors the charging of time capacitor in astable and monostable circuit.
	\item \textbf{Reset Input:} When it is less than 0.7 V, it makes the output low(0 V), overriding other inputs. When it is unnecessary it should be connected to the source voltage.
	\item \textbf{Control Input:} If there is a need to change the threshold voltage which is normally set to $2 V_S/3$, 	usually this is connected to 0 V with a very low capacitor of 0.01 $\mu$F to avoid electrical noise.
	\item \textbf{Discharge Pin: }
\end{enumerate}

\subsection{555/ 556 Astable Circuit:}
Time period ($T$) of the square wave is the time for one complete cycle. And frequency ($f$) of the wave is no. of complete cycles per second.

\begin{equation*}
	\begin{split}
		&T = 0.7 (R_1 + 2 R_2) C_1 \\
		&f = \frac{1.4}{(R_1 + 2 R_2) C_1}
	\end{split}
	\end{equation*}

Time period can be split into two part, when the output is high, \textbf{mark time} ($T_m$) and when the output is low,  \textbf{space time} ($T_s$).

\begin{equation*}
	\begin{split}
		&T_m = 0.7 (R_1 + R_2) C_1 \\
		&T_s = 0.7 R_2 C_1
	\end{split}.
\end{equation*}

%=========================================================================================
%									Data and Analysis									  
%=========================================================================================

\section{Data and Analysis}
\begin{table}[H]
		\centering
		\caption{Experimental and Theoretical Time Constants}
		\label{tab:time_constants}
		\begin{tabular}{
				S[table-format=1.3]
				S[table-format=1.0]
				S[table-format=3.0]
				S[table-format=1.1e1]
				S[table-format=3.1e1]
				S[table-format=1.1e1]
				S[table-format=1.1e1]
				S[table-format=2.1e1]
				S[table-format=2.2]
			}
			\toprule
			\toprule
			{$C_1$(\si{\micro\farad})} & {$R_1$} & {$R_2$} & {$f_{\text{theo}}$(\si{\kilo\hertz})} & {$T_m$(\si{\micro\second})} & {$T_s$(\si{\pico\second})} & {$T$(\si{\pico\second})} & {$f_{\text{expt}}$(\si{\kilo\hertz})} & {Error(\%)} \\
			\midrule
			\midrule 
			0.001 & 1 & 10 & 6.667e1 & 9.5e0 & 8.5e0 & 1.8e1 & 5.556e1 & 16.67 \\
			0.001 & 10 & 100 & 6.667e0 & 8.4e1 & 7.4e1 & 1.58e2 & 6.329e0 & 5.06 \\
			0.001 & 100 & 1000 & 6.667e-1 & 7.8e2 & 7.0e2 & 1.480e3 & 6.757e-1 & 1.35 \\
			0.01 & 1 & 10 & 6.667e0 & 7.6e1 & 7.0e1 & 1.46e2 & 6.849e0 & 2.74 \\
			0.01 & 10 & 100 & 6.667e-1 & 8.2e2 & 7.8e2 & 1.600e3 & 6.250e-1 & 6.25 \\
			0.01 & 100 & 1000 & 6.667e-2 & 8.4e3 & 7.6e3 & 1.600e4 & 6.250e-2 & 6.25 \\
			0.1 & 1 & 10 & 6.667e-1 & 6.4e2 & 6.0e2 & 1.280e3 & 7.813e-1 & 17.19 \\
			0.1 & 10 & 100 & 6.667e-2 & 7.2e3 & 6.8e3 & 1.400e4 & 7.143e-2 & 7.14 \\
			0.1 & 100 & 1000 & 6.667e-3 & 7.4e4 & 7.0e4 & 1.440e5 & 6.944e-3 & 4.16 \\
			1 & 1 & 10 & 6.667e-2 & 7.0e3 & 6.4e3 & 1.340e4 & 7.463e-2 & 11.94 \\
			1 & 10 & 100 & 6.667e-3 & 7.2e4 & 6.6e4 & 1.380e5 & 7.246e-3 & 8.69 \\
			1 & 100 & 1000 & 6.667e-4 & 7.0e5 & 6.4e5 & 1.340e6 & 7.463e-4 & 11.90 \\
			10 & 1 & 10 & 6.667e-3 & 7.4e4 & 6.8e4 & 1.420e5 & 7.042e-3 & 5.63 \\
			10 & 10 & 100 & 6.667e-4 & 8.0e5 & 7.2e5 & 1.520e5 & 6.579e-4 & 1.30 \\
			10 & 100 & 1000 & 6.667e-5 & 7.5e6 & 6.0e6 & 1.350e7 & 7.407e-5 & 11.00 \\
			\bottomrule
			\bottomrule
		\end{tabular}
	\end{table}
\end{document}
