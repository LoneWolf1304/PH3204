\documentclass{scrartcl}
\usepackage{pgfplots}
\usepackage{makecell}
\usepackage{multirow} % For merged cells
\usepackage{booktabs} % For better table formatting
\usepackage{calc}
\usepackage{Style_File}
\usepackage{circuitikz}
\usepackage{fancyhdr}
\usepackage{array}
\newcolumntype{P}[1]{>{\centering\arraybackslash}p{#1}}
% Recommended preamble:
\usetikzlibrary{arrows.meta}
\usetikzlibrary{backgrounds}
\usepgfplotslibrary{patchplots}
\usepgfplotslibrary{fillbetween}
\pgfplotsset{%
    layers/standard/.define layer set={%
        background,axis background,axis grid,axis ticks,axis lines,axis tick labels,pre main,main,axis descriptions,axis foreground%
    }{
        grid style={/pgfplots/on layer=axis grid},%
        tick style={/pgfplots/on layer=axis ticks},%
        axis line style={/pgfplots/on layer=axis lines},%
        label style={/pgfplots/on layer=axis descriptions},%
        legend style={/pgfplots/on layer=axis descriptions},%
        title style={/pgfplots/on layer=axis descriptions},%
        colorbar style={/pgfplots/on layer=axis descriptions},%
        ticklabel style={/pgfplots/on layer=axis tick labels},%
        axis background@ style={/pgfplots/on layer=axis background},%
        3d box foreground style={/pgfplots/on layer=axis foreground},%
    },
}


\setlength{\headheight}{0.75in}
\setlength{\oddsidemargin}{0in}
        \setlength{\evensidemargin}{0in}
        \setlength{\textwidth}{6.5in}
        \setlength{\headwidth}{7.3in}
        \setlength{\textheight}{8.75in}
        \rfoot{\thepage}
        \renewcommand{\headrulewidth}{0pt} % Remove the header line
        \renewcommand{\footrulewidth}{0pt}
\fancyhead[L,C]{}
\fancyhead[L]{PH3204: Experiment 4}
\fancyhead[R]{0\thepage}
\fancyhead[C]{ Boolean Algebra}
\fancyfoot[C]{0\thepage}
\fancyfoot[R,L]{}
\pagestyle{fancy}
\renewcommand{\headrulewidth}{0.4pt}


\usepackage{siunitx}
\usepackage{longtable} 
\usepackage[left = 0.7in,
right = 0.7in,
bottom = 0.9in,
top = 0.9in,
a4paper]{geometry}

\title{
        \Large\textsc{Experiment 03: }
        \huge\textbf{Study of Boolean Algebra} \\
}


\author{{\Large Sagnik Seth} -\   \texttt{22MS026}\\ ({\small Subgroup - A7}) }
\date{}


\begin{document}
\maketitle
\section{Aim}
To study the basic laws of Boolean Algebra and verify them using implementation of logic gates using Integrated Circuits.
\section{Theory}
\section{Verification of Truth Tables and Observations}
\begin{figure}[H]
    \centering
    \begin{circuitikz}
    % Draw the AND gate
    \node[ieeestd and port, fill=cyan!20] (and) at (0,0) {};

    % Draw the OR gate
    \node[ieeestd or port, fill=cyan!20] (or) at (0,-4) {};

    % Draw the NAND gate
    \node[ieeestd nand port, fill=cyan!20] (nand) at (3,-2) {};

    % Draw the NOR gate
    \node[ieeestd nor port, fill=cyan!20] (nor) at (6,-1) {};

    % Draw the XOR gate
    \node[ieeestd xor port, fill=cyan!20] (xor) at (6,-3.7) {};

    % Draw the final AND gate
    \node[ieeestd and port, fill=cyan!20] (final_and) at (9,-2) {};

    % Connect the AND gate to the NAND gate
    \draw (and.in 2) -- ++(-0,-1) |- (nand.in 1);
    \draw (and.out) -- ++(0.5,0) |- (nor.in 1);

    % Connect the OR gate to the NAND gate
    \draw (or.out) -- ++(0.5,0) |- (nand.in 2);
    \draw (or.out) -- ++(0.5,0) |- (xor.in 2);

    % Connect the NAND gate to the NOR gate
    \draw (nand.out) -- ++(0.5,0) |- (nor.in 2);

    % Connect the NAND gate to the XOR gate
    \draw (nand.out) -- ++(0.5,0) |- (xor.in 1);

    % Connect the NOR gate to the final AND gate
    \draw (nor.out) -- ++(0.5,0) |- (final_and.in 1);

    % Connect the XOR gate to the final AND gate
    \draw (xor.out) -- ++(0.5,0) |- (final_and.in 2);

    % Label inputs and outputs
    \draw (and.in 1) -- ++(-1,0) node[left] {A};
    \draw (and.in 2) -- ++(-1,0) node[left] {B};
    \draw (or.in 1) -- ++(-1,0) node[left] {C};
    \draw (or.in 2) -- ++(-1,0) node[left] {D};
    \draw (final_and.out) -- ++(1,0) node[right] {Q};
\end{circuitikz}
    \caption{Logic Gate circuit diagram for for Example 1}
\end{figure}
\begin{table}[H]
    \centering
    \caption{Truth Table for \( Y = \overline{\mathrm{A}}\mathrm{B}(\mathrm{C + D}) \)}
    \vspace{0.2cm}
    \begin{tabular}{|c|c|c|c||c|}
    \hline
    \textbf{A} & \textbf{B} & \textbf{C} & \textbf{D} & \textbf{Y} \\
    \hline
    0 & 0 & 0 & 0 & 0 \\
    0 & 0 & 0 & 1 & 0 \\
    0 & 0 & 1 & 0 & 0 \\
    0 & 0 & 1 & 1 & 0 \\
    0 & 1 & 0 & 0 & 0 \\
    0 & 1 & 0 & 1 & 1 \\
    0 & 1 & 1 & 0 & 1 \\
    0 & 1 & 1 & 1 & 1 \\
    1 & 0 & 0 & 0 & 0 \\
    1 & 0 & 0 & 1 & 0 \\
    1 & 0 & 1 & 0 & 0 \\
    1 & 0 & 1 & 1 & 0 \\
    1 & 1 & 0 & 0 & 0 \\
    1 & 1 & 0 & 1 & 0 \\
    1 & 1 & 1 & 0 & 0 \\
    1 & 1 & 1 & 1 & 0 \\
    \hline
    \end{tabular}
    \end{table}
\begin{figure}[H]
    \centering
    \begin{circuitikz}
    % Draw the AND gate
    \node[ieeestd and port, fill=cyan!20] (and) at (0,0) {};

    % Draw the OR gate
    \node[ieeestd nand port, fill=cyan!20] (nand) at (0,-4) {};

    % Draw the NAND gate
    \node[ieeestd xor port, fill=cyan!20] (xor) at (3,-2) {};

    % Draw the NOR gate
    \node[ieeestd nor port, fill=cyan!20] (nor) at (6,-1) {};

    % Draw the XOR gate
    \node[ieeestd and port, fill=cyan!20] (final_and) at (6,-3.72) {};

    % Draw the final AND gate
    \node[ieeestd or port, fill=cyan!20] (or) at (9,-2) {};

    % Connect the AND gate to the NAND gate
    \draw (and.in 2) -- ++(-0,-1) |- (xor.in 1);
    \draw (and.out) -- ++(0.5,0) |- (nor.in 1);

    % Connect the OR gate to the NAND gate
    \draw (nand.out) -- ++(0.5,0) |- (xor.in 2);
    \draw (nand.out) -- ++(0.5,0) |- (final_and.in 2);

    % Connect the NAND gate to the NOR gate
    \draw (xor.out) -- ++(0.5,0) |- (nor.in 2);

    % Connect the NAND gate to the XOR gate
    \draw (xor.out) -- ++(0.5,0) |- (final_and.in 1);

    % Connect the NOR gate to the final AND gate
    \draw (final_and.out) -- ++(0.5,0) |- (or.in 2);

    % Connect the XOR gate to the final AND gate
    \draw (nor.out) -- ++(0.5,0) |- (or.in 1);

    % Label inputs and outputs
    \draw (and.in 1) -- ++(-1,0) node[left] {A};
    \draw (and.in 2) -- ++(-1,0) node[left] {B};
    \draw (nand.in 1) -- ++(-1,0) node[left] {C};
    \draw (nand.in 2) -- ++(-1,0) node[left] {D};
    \draw (or.out) -- ++(1,0) node[right] {Q};
\end{circuitikz}
    \caption{Logic Gate circuit diagram for for Example 2}

\end{figure}
\begin{table}[H]
    \centering
    \caption{Truth Table for \( Y = \mathrm{\overline{A}\,\overline{C} + \overline{A}\,\overline{D} + \overline{B}} \)}
    \vspace{0.2cm}
    \begin{tabular}{|c|c|c|c||c|}
    \hline
    \textbf{A} & \textbf{B} & \textbf{C} & \textbf{D} & \textbf{Y} \\
    \hline
    0 & 0 & 0 & 0 & 1 \\
    0 & 0 & 0 & 1 & 1 \\
    0 & 0 & 1 & 0 & 1 \\
    0 & 0 & 1 & 1 & 1 \\
    0 & 1 & 0 & 0 & 1 \\
    0 & 1 & 0 & 1 & 1 \\
    0 & 1 & 1 & 0 & 1 \\
    0 & 1 & 1 & 1 & 0 \\
    1 & 0 & 0 & 0 & 1 \\
    1 & 0 & 0 & 1 & 1 \\
    1 & 0 & 1 & 0 & 1 \\
    1 & 0 & 1 & 1 & 1 \\
    1 & 1 & 0 & 0 & 1 \\
    1 & 1 & 0 & 1 & 0 \\
    1 & 1 & 1 & 0 & 0 \\
    1 & 1 & 1 & 1 & 0 \\
    \hline
    \end{tabular}
    \end{table}
\begin{figure}[H]
    \centering
    \begin{circuitikz}
    % Draw the AND gate
    \node[american nor port, fill=cyan!20] (nor) at (0,0) {};

    % Draw the OR gate
    \node[and port, fill=cyan!20] (and) at (0,-4) {};

    % Draw the NAND gate
    \node[or port, fill=cyan!20] (or) at (3,-2) {};

    % Draw the NOR gate
    \node[american nand port, fill=cyan!20] (nand) at (6,-1) {};

    % Draw the XOR gate
    \node[and port, fill=cyan!20] (and2) at (6,-3.72) {};

    % Draw the final AND gate
    \node[american xor port, fill=cyan!20] (xor) at (9,-2) {};

    % Connect the AND gate to the NAND gate
    \draw (nor.in 2) -- ++(-0,-1) |- (or.in 1);
    \draw (nor.out) -- ++(0.5,0) |- (nand.in 1);

    % Connect the OR gate to the NAND gate
    \draw (and.out) -- ++(0.5,0) |- (or.in 2);
    \draw (and.out) -- ++(0.5,0) |- (and2.in 2);

    % Connect the NAND gate to the NOR gate
    \draw (or.out) -- ++(0.5,0) |- (nand.in 2);

    % Connect the NAND gate to the XOR gate
    \draw (or.out) -- ++(0.5,0) |- (and2.in 1);

    % Connect the NOR gate to the final AND gate
    \draw (nand.out) -- ++(0.5,0) |- (xor.in 1);

    % Connect the XOR gate to the final AND gate
    \draw (and2.out) -- ++(0.5,0) |- (xor.in 2);

    % Label inputs and outputs
    \draw (nor.in 1) -- ++(-1,0) node[left] {A};
    \draw (nor.in 2) -- ++(-1,0) node[left] {B};
    \draw (and.in 1) -- ++(-1,0) node[left] {C};
    \draw (and.in 2) -- ++(-1,0) node[left] {D};
    \draw (xor.out) -- ++(1,0) node[right] {Q};
\end{circuitikz}

    \caption{Logic Gate circuit diagram for for Example 3}
\end{figure}
\begin{table}[H]
    \centering
    \caption{Truth Table for \(\mathrm{ Y = \overline{A}\,\overline{B} + \overline{D} + \overline{C} }\)}
    \vspace{0.2cm}
    \begin{tabular}{|c|c|c|c||c|}
    \hline
    \textbf{A} & \textbf{B} & \textbf{C} & \textbf{D} & \textbf{Y} \\
    \hline
    0 & 0 & 0 & 0 & 1 \\
    0 & 0 & 0 & 1 & 1 \\
    0 & 0 & 1 & 0 & 1 \\
    0 & 0 & 1 & 1 & 1 \\
    0 & 1 & 0 & 0 & 1 \\
    0 & 1 & 0 & 1 & 1 \\
    0 & 1 & 1 & 0 & 1 \\
    0 & 1 & 1 & 1 & 0 \\
    1 & 0 & 0 & 0 & 1 \\
    1 & 0 & 0 & 1 & 1 \\
    1 & 0 & 1 & 0 & 1 \\
    1 & 0 & 1 & 1 & 0 \\
    1 & 1 & 0 & 0 & 1 \\
    1 & 1 & 0 & 1 & 1 \\
    1 & 1 & 1 & 0 & 1 \\
    1 & 1 & 1 & 1 & 0 \\
    \hline
    \end{tabular}
    \end{table}
\section{Sources of Error}
\section{Discussion and Conclusion}
\end{document}