\documentclass{scrartcl}
\usepackage{pgfplots}
\usepackage{makecell}
\usepackage{multirow} % For merged cells
\usepackage{booktabs} % For better table formatting
\usepackage{calc}
\usepackage{Style_File}
\usepackage{circuitikz}
\usepackage{fancyhdr}
\usepackage{array}
\newcolumntype{P}[1]{>{\centering\arraybackslash}p{#1}}
% Recommended preamble:
\usetikzlibrary{arrows.meta}
\usetikzlibrary{backgrounds}
\usepgfplotslibrary{patchplots}
\usepgfplotslibrary{fillbetween}
\pgfplotsset{%
    layers/standard/.define layer set={%
        background,axis background,axis grid,axis ticks,axis lines,axis tick labels,pre main,main,axis descriptions,axis foreground%
    }{
        grid style={/pgfplots/on layer=axis grid},%
        tick style={/pgfplots/on layer=axis ticks},%
        axis line style={/pgfplots/on layer=axis lines},%
        label style={/pgfplots/on layer=axis descriptions},%
        legend style={/pgfplots/on layer=axis descriptions},%
        title style={/pgfplots/on layer=axis descriptions},%
        colorbar style={/pgfplots/on layer=axis descriptions},%
        ticklabel style={/pgfplots/on layer=axis tick labels},%
        axis background@ style={/pgfplots/on layer=axis background},%
        3d box foreground style={/pgfplots/on layer=axis foreground},%
    },
}


\setlength{\headheight}{0.75in}
\setlength{\oddsidemargin}{0in}
        \setlength{\evensidemargin}{0in}
        \setlength{\textwidth}{6.5in}
        \setlength{\headwidth}{7.3in}
        \setlength{\textheight}{8.75in}
        \rfoot{\thepage}
        \renewcommand{\headrulewidth}{0pt} % Remove the header line
        \renewcommand{\footrulewidth}{0pt}
\fancyhead[L,C]{}
\fancyhead[L]{PH3204: Experiment 4}
\fancyhead[R]{0\thepage}
\fancyhead[C]{ Boolean Algebra}
\fancyfoot[C]{0\thepage}
\fancyfoot[R,L]{}
\pagestyle{fancy}
\renewcommand{\headrulewidth}{0.4pt}


\usepackage{siunitx}
\usepackage{longtable} 
\usepackage[left = 0.7in,
right = 0.7in,
bottom = 0.9in,
top = 0.9in,
a4paper]{geometry}

\title{
        \Large\textsc{Experiment 04: }
        \huge\textbf{Study of Boolean Algebra} \\
}


\author{{\Large Sagnik Seth} -\   \texttt{22MS026}\\ ({\small Subgroup - A7}) }
\date{}


\begin{document}
\maketitle
\section{Aim}
To study the basic laws of Boolean Algebra and verify them using implementation of logic gates using Integrated Circuits.
\section{Theory}
Boolean Algebra deals with logical operations of binary variables. A Boolean function $f$ is generally defined as $f: \{0,1\}^k \rightarrow \{0,1\} $, that is, it takes input from $k$ cartesian products of the set $\{0,1\}$ and returns a single output which can be 0 or 1. The basic boolean functions are: 
\begin{itemize}
    \item \textbf{NOT Gate}\\[0.3cm]
    It is a unary operator which takes a single input and returns the negation of the input. It is represented by the symbol $\overline{\mathrm{A}}$. If input is 1, it return 0 and if input is 0, it returns 1.
    \begin{figure}[H]
        \centering
        \begin{circuitikz}
            \draw (0,0) node[ieeestd not port, fill=cyan!40!white] (mynot) {};
            \draw (mynot.in) -- ++(-1,0) node[left] {A};
            \draw (mynot.out) -- ++(1,0) node[right] {Y};
        \end{circuitikz}
        \caption{Representation of NOT Gate}
    \end{figure}
    \item \textbf{AND Gate}\\[0.3cm]
    It is a binary operator which takes two inputs and returns the logical AND of the inputs. It is represented by the symbol $\mathrm{A}\cdot \mathrm{B}$. It returns 1 only if both inputs are 1, otherwise it returns 0.
    \begin{figure}[H]
        \centering
        \begin{circuitikz}
            \draw (0,0) node[ieeestd and port, fill=cyan!40!white] (myand) {};
            \draw (myand.in 1) -- ++(-1,0) node[left] {A};
            \draw (myand.in 2) -- ++(-1,0) node[left] {B};
            \draw (myand.out) -- ++(1,0) node[right] {Y};
        \end{circuitikz}
        \caption{Representation of AND Gate}
    \end{figure}
    \item  \textbf{OR Gate}\\[0.3cm]
    It is a binary operator which takes two inputs and returns the logical OR of the inputs. It is represented by the symbol $\mathrm{A} + \mathrm{B}$. It returns 1 if any of the inputs is 1, otherwise it returns 0.
    \begin{figure}[H]
        \centering
        \begin{circuitikz}
            \draw (0,0) node[ieeestd or port, fill=cyan!40!white] (myor) {};
            \draw (myor.in 1) -- ++(-1,0) node[left] {A};
            \draw (myor.in 2) -- ++(-1,0) node[left] {B};
            \draw (myor.out) -- ++(1,0) node[right] {Y};
        \end{circuitikz}
        \caption{Representation of OR Gate}
    \end{figure}
    \item \textbf{XOR Gate}\\[0.3cm]
    It is a binary operator which takes two inputs and returns the logical XOR of the inputs. It is represented by the symbol $\mathrm{A} \oplus \mathrm{B}$. It can be showed that the expression for XOR is equivalent to $\mathrm{\overline{A}B+A\overline{B}}$. It returns 1 if the inputs are different, otherwise it returns 0.
    \begin{figure}[H]
        \centering
        \begin{circuitikz}
            \draw (0,0) node[ieeestd xor port, fill=cyan!40!white] (myxor) {};
            \draw (myxor.in 1) -- ++(-1,0) node[left] {A};
            \draw (myxor.in 2) -- ++(-1,0) node[left] {B};
            \draw (myxor.out) -- ++(1,0) node[right] {Y};
        \end{circuitikz}
        \caption{Representation of XOR Gate}
    \end{figure}
    \item \textbf{NAND Gate}\\[0.3cm]
    It is a binary operator which takes two inputs and returns the logical NAND of the inputs. It is represented by the symbol $\overline{\mathrm{A} \mathrm{B}}$. Thus, it is a composition of AND and NOT operations. From de Morgan's law, it can be shown that NAND is equivalent to $\overline{\mathrm{A}} + \overline{\mathrm{B}}$. 
    \begin{figure}[H]
        \centering
        \begin{circuitikz}
            \draw (0,0) node[ieeestd nand port, fill=cyan!40!white] (mynand) {};
            \draw (mynand.in 1) -- ++(-1,0) node[left] {A};
            \draw (mynand.in 2) -- ++(-1,0) node[left] {B};
            \draw (mynand.out) -- ++(1,0) node[right] {Y};
        \end{circuitikz}
        \caption{Representation of NAND Gate}
    \end{figure}
    \item \textbf{NOR Gate}
    It is a binary operator which takes two inputs and returns the logical NOR of the inputs. It is represented by the symbol $\overline{\mathrm{A} + \mathrm{B}}$. Thus, it is a composition of OR and NOT operations. From de Morgan's law, it can be shown that NOR is equivalent to $\overline{\mathrm{A}} \cdot \overline{\mathrm{B}}$.
    \begin{figure}[H]
        \centering
        \begin{circuitikz}
            \draw (0,0) node[ieeestd nor port, fill=cyan!40!white] (mynor) {};
            \draw (mynor.in 1) -- ++(-1,0) node[left] {A};
            \draw (mynor.in 2) -- ++(-1,0) node[left] {B};
            \draw (mynor.out) -- ++(1,0) node[right] {Y};
        \end{circuitikz}
        \caption{Representation of NOR Gate}
    \end{figure}
\end{itemize}
\begin{table}[H]
    \centering
    \label{tab:truth_table}
    \renewcommand{\arraystretch}{1.5} % Increase row height
    \begin{tabular}{|c|c||c|c|c|c|c|}
    \hline
    \textbf{A} & \textbf{B} & {$AB$} & {$A + B$} & {$\overline{AB}$} & {$\overline{A+B}$} & {$A \oplus B$} \\
    \hline
    0 & 0 & 0 & 0 & 1 & 1 & 0 \\
    0 & 1 & 0 & 1 & 1 & 0 & 1 \\
    1 & 0 & 0 & 1 & 1 & 0 & 1 \\
    1 & 1 & 1 & 1 & 0 & 0 & 0 \\
    \hline
    \end{tabular}
    \renewcommand{\arraystretch}{1} % Reset to default
    \caption{Truth Table for the above logical operations}
\end{table}
The logic gates can be implemented using Integrated Circuit (IC). They generally have 14 pins and multiple gates of the same kind are implemented on the same IC chip. The ICs used in this experiment are:
\begin{itemize}
    \item  IC 7408 (AND Gate)
    \item IC 7432 (OR Gate)
    \item IC 7404 (NOT Gate)
    \item IC 7400 (NAND Gate)
    \item IC 7402 (NOR Gate)
    \item IC 7486 (XOR Gate)

\end{itemize}
 Each of these chips have a V\textsubscript{CC} and Ground terminal which connects it to the main circuit. \\[0.3cm]
 The logic gates forms basic structure of digital electronics characterised by discrete bands of analog levels, contrary to the continuous range in analog electronics. The zero state represents the \textbf{OFF} state while the one state represents the \textbf{ON} state. 
 
 \section{Verification of Truth Tables and Observations}
\textbf{Example 1:}
\begin{figure}[H]
    \centering
    \begin{circuitikz}
    % Draw the AND gate
    \node[ieeestd and port, fill=cyan!20] (and) at (0,0) {};

    % Draw the OR gate
    \node[ieeestd or port, fill=cyan!20] (or) at (0,-4) {};

    % Draw the NAND gate
    \node[ieeestd nand port, fill=cyan!20] (nand) at (3,-2) {};

    % Draw the NOR gate
    \node[ieeestd nor port, fill=cyan!20] (nor) at (6,-1) {};

    % Draw the XOR gate
    \node[ieeestd xor port, fill=cyan!20] (xor) at (6,-3.7) {};

    % Draw the final AND gate
    \node[ieeestd and port, fill=cyan!20] (final_and) at (9,-2) {};

    % Connect the AND gate to the NAND gate
    \draw (and.in 2) -- ++(-0,-1) |- (nand.in 1);
    \draw (and.out) -- ++(0.5,0) |- (nor.in 1);

    % Connect the OR gate to the NAND gate
    \draw (or.out) -- ++(0.5,0) |- (nand.in 2);
    \draw (or.out) -- ++(0.5,0) |- (xor.in 2);

    % Connect the NAND gate to the NOR gate
    \draw (nand.out) -- ++(0.5,0) |- (nor.in 2);

    % Connect the NAND gate to the XOR gate
    \draw (nand.out) -- ++(0.5,0) |- (xor.in 1);

    % Connect the NOR gate to the final AND gate
    \draw (nor.out) -- ++(0.5,0) |- (final_and.in 1);

    % Connect the XOR gate to the final AND gate
    \draw (xor.out) -- ++(0.5,0) |- (final_and.in 2);

    % Label inputs and outputs
    \draw (and.in 1) -- ++(-1,0) node[left] {A};
    \draw (and.in 2) -- ++(-1,0) node[left] {B};
    \draw (or.in 1) -- ++(-1,0) node[left] {C};
    \draw (or.in 2) -- ++(-1,0) node[left] {D};
    \draw (final_and.out) -- ++(1,0) node[right] {Q};
\end{circuitikz}
    \caption{Logic Gate circuit diagram for for Example 1}
\end{figure}
The analytic expression for the above circuit was found out to be: $$\mathrm{Q = \overline{A}B(C+D)}$$
\begin{table}[H]
    \centering
    \caption{Truth Table for \( \mathrm{Q} = \overline{\mathrm{A}}\mathrm{B}(\mathrm{C + D}) \)}
    \vspace{0.2cm}
    \begin{tabular}{|c|c|c|c||c|}
    \hline
    \textbf{A} & \textbf{B} & \textbf{C} & \textbf{D} & \textbf{Y} \\
    \hline
    0 & 0 & 0 & 0 & 0 \\
    0 & 0 & 0 & 1 & 0 \\
    0 & 0 & 1 & 0 & 0 \\
    0 & 0 & 1 & 1 & 0 \\
    0 & 1 & 0 & 0 & 0 \\
    0 & 1 & 0 & 1 & 1 \\
    0 & 1 & 1 & 0 & 1 \\
    0 & 1 & 1 & 1 & 1 \\
    1 & 0 & 0 & 0 & 0 \\
    1 & 0 & 0 & 1 & 0 \\
    1 & 0 & 1 & 0 & 0 \\
    1 & 0 & 1 & 1 & 0 \\
    1 & 1 & 0 & 0 & 0 \\
    1 & 1 & 0 & 1 & 0 \\
    1 & 1 & 1 & 0 & 0 \\
    1 & 1 & 1 & 1 & 0 \\
    \hline
    \end{tabular}
    \end{table}
    \noindent
    We created the circuit on the breadboard using the integrated circuits and connected all the components. Then, using an LED Bulb, we verified the above truth table for the circuit shown. For all configurations of the 4 inputs, the output in the circuit matched with the theoretical truth table, that is, the LED bulb glowed when the output was 1 and it did not glow when the output was 0. \\[0.2cm]
    \textbf{Example 2:}  
    \begin{figure}[H]
    \centering
    \begin{circuitikz}
    % Draw the AND gate
    \node[ieeestd and port, fill=cyan!20] (and) at (0,0) {};

    % Draw the OR gate
    \node[ieeestd nand port, fill=cyan!20] (nand) at (0,-4) {};

    % Draw the NAND gate
    \node[ieeestd xor port, fill=cyan!20] (xor) at (3,-2) {};

    % Draw the NOR gate
    \node[ieeestd nor port, fill=cyan!20] (nor) at (6,-1) {};

    % Draw the XOR gate
    \node[ieeestd and port, fill=cyan!20] (final_and) at (6,-3.72) {};

    % Draw the final AND gate
    \node[ieeestd or port, fill=cyan!20] (or) at (9,-2) {};

    % Connect the AND gate to the NAND gate
    \draw (and.in 2) -- ++(-0,-1) |- (xor.in 1);
    \draw (and.out) -- ++(0.5,0) |- (nor.in 1);

    % Connect the OR gate to the NAND gate
    \draw (nand.out) -- ++(0.5,0) |- (xor.in 2);
    \draw (nand.out) -- ++(0.5,0) |- (final_and.in 2);

    % Connect the NAND gate to the NOR gate
    \draw (xor.out) -- ++(0.5,0) |- (nor.in 2);

    % Connect the NAND gate to the XOR gate
    \draw (xor.out) -- ++(0.5,0) |- (final_and.in 1);

    % Connect the NOR gate to the final AND gate
    \draw (final_and.out) -- ++(0.5,0) |- (or.in 2);

    % Connect the XOR gate to the final AND gate
    \draw (nor.out) -- ++(0.5,0) |- (or.in 1);

    % Label inputs and outputs
    \draw (and.in 1) -- ++(-1,0) node[left] {A};
    \draw (and.in 2) -- ++(-1,0) node[left] {B};
    \draw (nand.in 1) -- ++(-1,0) node[left] {C};
    \draw (nand.in 2) -- ++(-1,0) node[left] {D};
    \draw (or.out) -- ++(1,0) node[right] {Q};
\end{circuitikz}
    \caption{Logic Gate circuit diagram for for Example 2}

\end{figure}
The analytic expression for the above circuit was found out to be: 
$$\mathrm{Q = \overline{A}\ \overline{C}+ \overline{A}\ \overline{D} + \overline{B}}$$

\begin{table}[H]
    \centering
    \caption{Truth Table for \( \mathrm{Q} = \mathrm{\overline{A}\,\overline{C} + \overline{A}\,\overline{D} + \overline{B}} \)}
    \vspace{0.2cm}
    \begin{tabular}{|c|c|c|c||c|}
    \hline
    \textbf{A} & \textbf{B} & \textbf{C} & \textbf{D} & \textbf{Y} \\
    \hline
    0 & 0 & 0 & 0 & 1 \\
    0 & 0 & 0 & 1 & 1 \\
    0 & 0 & 1 & 0 & 1 \\
    0 & 0 & 1 & 1 & 1 \\
    0 & 1 & 0 & 0 & 1 \\
    0 & 1 & 0 & 1 & 1 \\
    0 & 1 & 1 & 0 & 1 \\
    0 & 1 & 1 & 1 & 0 \\
    1 & 0 & 0 & 0 & 1 \\
    1 & 0 & 0 & 1 & 1 \\
    1 & 0 & 1 & 0 & 1 \\
    1 & 0 & 1 & 1 & 1 \\
    1 & 1 & 0 & 0 & 1 \\
    1 & 1 & 0 & 1 & 0 \\
    1 & 1 & 1 & 0 & 0 \\
    1 & 1 & 1 & 1 & 0 \\
    \hline
    \end{tabular}
    \end{table}
    \noindent
    We created the circuit on the breadboard using the integrated circuits and connected all the components. Then, using an LED Bulb, we verified the above truth table for the circuit shown. For all configurations of the 4 inputs, the output in the circuit matched with the theoretical truth table, that is, the LED bulb glowed when the output was 1 and it did not glow when the output was 0. \\[0.2cm]
    \textbf{Example 3:}  
\begin{figure}[H]
    \centering
    \begin{circuitikz}
    % Draw the AND gate
    \node[american nor port, fill=cyan!20] (nor) at (0,0) {};

    % Draw the OR gate
    \node[and port, fill=cyan!20] (and) at (0,-4) {};

    % Draw the NAND gate
    \node[or port, fill=cyan!20] (or) at (3,-2) {};

    % Draw the NOR gate
    \node[american nand port, fill=cyan!20] (nand) at (6,-1) {};

    % Draw the XOR gate
    \node[and port, fill=cyan!20] (and2) at (6,-3.72) {};

    % Draw the final AND gate
    \node[american xor port, fill=cyan!20] (xor) at (9,-2) {};

    % Connect the AND gate to the NAND gate
    \draw (nor.in 2) -- ++(-0,-1) |- (or.in 1);
    \draw (nor.out) -- ++(0.5,0) |- (nand.in 1);

    % Connect the OR gate to the NAND gate
    \draw (and.out) -- ++(0.5,0) |- (or.in 2);
    \draw (and.out) -- ++(0.5,0) |- (and2.in 2);

    % Connect the NAND gate to the NOR gate
    \draw (or.out) -- ++(0.5,0) |- (nand.in 2);

    % Connect the NAND gate to the XOR gate
    \draw (or.out) -- ++(0.5,0) |- (and2.in 1);

    % Connect the NOR gate to the final AND gate
    \draw (nand.out) -- ++(0.5,0) |- (xor.in 1);

    % Connect the XOR gate to the final AND gate
    \draw (and2.out) -- ++(0.5,0) |- (xor.in 2);

    % Label inputs and outputs
    \draw (nor.in 1) -- ++(-1,0) node[left] {A};
    \draw (nor.in 2) -- ++(-1,0) node[left] {B};
    \draw (and.in 1) -- ++(-1,0) node[left] {C};
    \draw (and.in 2) -- ++(-1,0) node[left] {D};
    \draw (xor.out) -- ++(1,0) node[right] {Q};
\end{circuitikz}

    \caption{Logic Gate circuit diagram for for Example 3}
\end{figure}
The analytic expression for the above circuit was found out to be: 
$$\mathrm{Q = \overline{A}\ \overline{B}+ \overline{C} + \overline{D}}$$
\begin{table}[H]
    \centering
    \caption{Truth Table for \(\mathrm{ \mathrm{Q} = \overline{A}\,\overline{B} + \overline{D} + \overline{C} }\)}
    \vspace{0.2cm}
    \begin{tabular}{|c|c|c|c||c|}
    \hline
    \textbf{A} & \textbf{B} & \textbf{C} & \textbf{D} & \textbf{Y} \\
    \hline
    0 & 0 & 0 & 0 & 1 \\
    0 & 0 & 0 & 1 & 1 \\
    0 & 0 & 1 & 0 & 1 \\
    0 & 0 & 1 & 1 & 1 \\
    0 & 1 & 0 & 0 & 1 \\
    0 & 1 & 0 & 1 & 1 \\
    0 & 1 & 1 & 0 & 1 \\
    0 & 1 & 1 & 1 & 0 \\
    1 & 0 & 0 & 0 & 1 \\
    1 & 0 & 0 & 1 & 1 \\
    1 & 0 & 1 & 0 & 1 \\
    1 & 0 & 1 & 1 & 0 \\
    1 & 1 & 0 & 0 & 1 \\
    1 & 1 & 0 & 1 & 1 \\
    1 & 1 & 1 & 0 & 1 \\
    1 & 1 & 1 & 1 & 0 \\
    \hline
    \end{tabular}
    \end{table}
    \noindent
    We created the circuit on the breadboard using the integrated circuits and connected all the components. Then, using an LED Bulb, we verified the above truth table for the circuit shown. For all configurations of the 4 inputs, the output in the circuit matched with the theoretical truth table, that is, the LED bulb glowed when the output was 1 and it did not glow when the output was 0. 
\section{Sources of Error}
\begin{itemize}
    \item The ICs used may be faulty or damaged. We encountered an IC which did not gave an expected output. 
    \item We had to use a resistance with the LED bulb, otherwise it would burn out due to excessive current. 
\end{itemize}
\section{Discussion and Conclusion}
We verified the truth tables for the above circuits using the logic gates and ICs. The output of the circuit matched with the theoretical truth table for all configurations of the inputs. 
\end{document}