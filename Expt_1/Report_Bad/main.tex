\documentclass[12pt]{article}

\usepackage{amsmath}
\usepackage{graphicx}
\usepackage{svg}
\usepackage{longtable} 
\usepackage{circuitikz}
\usepackage[margin=1.2in]{geometry}


\begin{document}
\title{Sub-Group: A-7 \\ Experiment 1: Study of Zener Diode and IC7805}


\author{Sayan Karmakar \\22MS163 }
\date{}
\maketitle

\section{Aim}
Aim of this experiment is to
\begin{itemize}
    \item Study voltage regulation of zener diode.
    \item Study voltage regulation of IC7805.
\end{itemize}

\section{Theory}
\subsection{Zener Diode:}
Zener diode is a general purpose diode that acts as a normal diode when connected to the circuit in forward bias but when it is connected in reverse bias it acts as a voltage regulator for a wide range of currents.

In forward bias a zener diode has the same I-V characteristics as a  regular diode i.e. there is a knee voltage after which current starts increasing exponentially with voltage. In reverse bias initially for any voltage the current remains constant. This is called saturation current. But increasing the voltage further causes it to reach the reverse bias breakdown region. When breakdown voltage is reached, even a slight change in voltage can cause a very large increase in current through the zener diode. This change in saturation current to breakdown is smooth but after this the current increases rapidly. Because of this, voltage remains almost constant throughout this increase in current. This is the reason, zener diode is used as a voltage regulator.
As the current increases, the power dissipation in the zener diode also increases. So for very current, due to excessive heat the zener diode can be damaged. So they have a maximum current rating. And it should be used below that current level.

Purpose of a voltage regulator is to supply constant voltage to any load present in the circuit in spite of fluctuations in the supply voltage or load current. To do this, the zener diode is used in parallel to the load. So if the zener diode reaches breakdown region and the current through the zener diode is less than maximum current then voltage across the diode will remain almost constant and the same thing will happen to the load.

\begin{figure}[!ht]
    \centering
    \resizebox{0.5\textwidth}{!}{%
    \begin{circuitikz}
    \tikzstyle{every node}=[font=\normalsize]
    \draw (2.5,14.75) to[battery ] (2.5,10.25);
    \draw (2.5,14.75) to[R] (5.25,14.75);
    \node at (7,14.75) [circ] {};
    \node at (7,14.75) [circ] {};
    \draw (2.5,10.25) to[short] (8.5,10.25);
    \draw (7,14.75) to[short] (8.5,14.75);
    \draw (8.5,14.75) to[R] (8.5,13);
    \draw (7,10.25) to[empty tunnel diode] (7,12.25);
    \draw (8.5,12) to[R] (8.5,10.25);
    \draw [->, >=Stealth] (8.25,11.75) -- (9,10.75);
    \draw  (6.25,14.75) circle (0.5cm);
    \draw  (7,13.5) circle (0.5cm);
    \draw  (8.5,12.5) circle (0.5cm);
    \draw (5.25,14.75) to[short] (5.75,14.75);
    \draw (6.75,14.75) to[short] (7.25,14.75);
    \draw (7,14.75) to[short] (7,14);
    \draw (7,13) to[short] (7,12);
    \node [font=\normalsize] at (6.25,14.75) {mA};
    \node [font=\normalsize] at (7,13.5) {mA};
    \node [font=\normalsize] at (8.5,12.5) {mA};
    \node [font=\normalsize] at (6,15.5) {$i_s$};
    \node [font=\normalsize] at (6.25,13.5) {$i_z$};
    \node [font=\normalsize] at (8,14.25) {$R_C$};
    \node [font=\normalsize] at (7.75,12.5) {$i_L$};
    \node [font=\normalsize] at (9,11.25) {$R_L$};
    \draw (8.5,14.75) to[short, -o] (9.75,14.75) ;
    \draw (8.5,10.25) to[short, -o] (9.75,10.25) ;
    \node [font=\normalsize] at (3.25,12.5) {$V_{in}$};
    \end{circuitikz}
    }%
    
    \label{fig:my_label}
    \end{figure}

\subsection{IC:}
Using a zener diode as a voltage regulator has a major drawback. Regulation is not perfect as the voltage is not really constant, it changes very slowly with the current. Also the maximum current that can flow through the diode without damaging it is quite low. So, tackle this, IC based voltage regulators are used. 

In this experiment, IC7805 is used. This has three pins. These pins are input, output, and ground. The last two digits of the number represent the voltage it regulates. While maximum current in zener diode is in the milliampere range, for IC7805 it is 1 ampere. It also has thermal overload protection and short circuit protection. The circuit diagram is shown below.

\begin{figure}[!ht]
    \centering
    \resizebox{0.5\textwidth}{!}{%
    \begin{circuitikz}
    \tikzstyle{every node}=[font=\normalsize]
    \draw (2.5,14.75) to[battery ] (2.5,10.25);
    \draw (2.5,10.25) to[short] (8.5,10.25);
    \draw (8.5,14.75) to[R] (8.5,13);
    \draw (8.5,12) to[R] (8.5,10.25);
    \draw [->, >=Stealth] (8.25,11.75) -- (9,10.75);
    \draw  (8.5,12.5) circle (0.5cm);
    \node [font=\normalsize] at (8.5,12.5) {mA};
    \node [font=\normalsize] at (8,14.25) {$R_C$};
    \node [font=\normalsize] at (7.75,12.5) {$i_L$};
    \node [font=\normalsize] at (9,11.25) {$R_L$};
    \draw (8.5,14.75) to[short, -o] (9.75,14.75) ;
    \draw (8.5,10.25) to[short, -o] (9.75,10.25) ;
    \node [font=\normalsize] at (3.25,12.5) {$V_{in}$};
    \draw (2.5,14.75) to[short] (4.5,14.75);
    \draw  (4.5,15.25) rectangle (6.75,14);
    \draw (6.75,14.75) to[short] (8.5,14.75);
    \draw (5.5,14) to[short] (5.5,10.25);
    \node [font=\normalsize] at (4,15) {In};
    \node [font=\normalsize] at (6.25,13.75) {Ground};
    \node [font=\normalsize] at (7.25,15) {Out};
    \node [font=\normalsize] at (5.75,14.5) {\textbf{IC7805}};
    \end{circuitikz}
    }%
    
    \label{fig:my_label}
    \end{figure}
\section{Data and Calculation}
For both zener diode and IC7805 there are two types of regulation - line regulation and load regulation. In line regulation the load resistance $R_l$ and series resistance $R_s$ is kept fixed and the input voltage is varied. And in load regulation, input voltage $V_m$ remains constant but the load resistance $R_l$ is varied. In this experiment data collection has been done for both the regulations in case of zener diode and IC7805.

\subsection{Zener Diode:}
\subsubsection{Line Regulation}

In case of line regulation for zener diode, load resistance was kept at $R_l = 1.1 \ \text{k}\Omega$
Variation of output volatge $V_o$, current across zener $I_z$ and current in the circuit $I_s$ with respest to the change in input voltage is shown in the following table:
\begin{longtable}{|l|l|l|l|}
        \hline
        $\mathrm{V_i}$ (V) & $\text{I}_{s}$ (mA) & $\text{I}_\text{z}$ (mA) & $\text{V}_\text{o}$ (V) \\ \hline \hline
        \endfirsthead
        \hline
        $\mathrm{V_i}$ (V) & $\text{I}_{s}$ (mA) & $\text{I}_\text{z}$ (mA) & $\text{V}_\text{o}$ (V) \\ \hline \hline
        \endhead
%        \hline
%        \endfoot
%        \hline
%        \endlastfoot
        0      & 0       & 0.00000       & 0.01    \\ \hline
        0.6    & 0       & 0.00000       & 0.4     \\ \hline
        1      & 0.12    & 0.00000      & 0.69    \\ \hline
        1.6    & 0.23    & 0.00000       & 1.02    \\ \hline
        2.1    & 0.35    & 0.00000      & 1.31    \\ \hline
        2.4    & 0.46    & 0.00000       & 1.52    \\ \hline
        3      & 0.12    & 0.00000       & 1.85    \\ \hline
        3.5    & 0.23    & 0.00000       & 2.16    \\ \hline
        3.9    & 0.58    & 0.00000      & 2.43    \\ \hline
        4.5    & 0.69    & 0.00000       & 2.77    \\ \hline
        5      & 0.81    & 0.00002       & 3.09    \\ \hline
        5.4    & 0.93    & 0.00008       & 3.31    \\ \hline
        6      & 1.05    & 0.00023       & 3.68    \\ \hline
        6.5    & 1.18    & 0.00040       & 3.92    \\ \hline
        7      & 1.27    & 0.00083       & 4.23    \\ \hline
        7.5    & 1.39    & 0.00161       & 4.57    \\ \hline
        8      & 1.51    & 0.00337       & 4.87    \\ \hline
        8.5    & 1.63    & 0.00637       & 5.16    \\ \hline
        8.9    & 1.74    & 0.01132       & 5.40    \\ \hline
        9.5    & 1.86    & 0.01666       & 5.71    \\ \hline
        10     & 1.98    & 0.03700       & 6.05    \\ \hline
        10.5   & 1.58    & 0.04000       & 6.35    \\ \hline
        11     & 2.11    & 0.10000       & 6.50    \\ \hline
        11.5   & 2.30    & 0.30000       & 6.53    \\ \hline
        11.9   & 2.50    & 0.50000       & 6.57    \\ \hline
        12.5   & 2.74    & 0.70000       & 6.55    \\ \hline
        13     & 3.01    & 1.00000       & 6.56    \\ \hline
        13.6   & 3.24    & 1.30000       & 6.57    \\ \hline
        13.9   & 3.41    & 1.40000       & 6.57    \\ \hline
        14.5   & 3.67    & 1.70000       & 6.57    \\ \hline
        15     & 3.89    & 1.90000       & 6.58    \\ \hline
\end{longtable}
We plotted \( \mathrm{I_s} \) and \( \mathrm{I_z} \) from this table.

\subsubsection{Load Regulation}
The load regulation of zener diode was studied in two cases. In the first case it was without $R_c$ and in the second case with $R_c$ set to 2.2 $\text{k}\Omega$.\\[0.3cm]
\textbf{Without $\mathrm{R_c}:$ }\\[0.3cm]
For the load regulation, we fixed the input voltage to $V_i = 15 \ \mathrm{V}$ and varied the load resistance using a potentiometer. The output voltage, current across zener diode and the load current was measured for each load resistance. The data is tabulated below:
\begin{longtable}{|l|l|l|l|}
\hline
$\mathrm{R_L}$ ($\mathrm{k}\Omega$) & $\mathrm{I_L}$ (mA) & $\text{I}_\text{z}$ (mA) & $\mathrm{V_o}$ (V) \\ \hline \hline
\endfirsthead
\hline
$\mathrm{R_L}$ ($\mathrm{k}\Omega$) & $\mathrm{I_L}$ (mA) & $\text{I}_\text{z}$  (mA) & $\mathrm{V_o}$ (V) \\ \hline \hline
\endhead
\hline
\endfoot
\hline
\endlastfoot
0.007       & 6.92      & 0       & 0.2       \\ \hline
0.057       & 6.76      & 0       & 0.5       \\ \hline
0.106       & 6.61      & 0       & 0.9       \\ \hline
0.153       & 6.5       & 0       & 1.2       \\ \hline
0.203       & 6.36      & 0       & 1.4       \\ \hline
0.249       & 6.25      & 0       & 1.7       \\ \hline
0.304       & 6.11      & 0       & 2.1       \\ \hline
0.354       & 6.02      & 0       & 2.3       \\ \hline
0.396       & 5.92      & 0       & 2.6       \\ \hline
0.452       & 5.78      & 0       & 2.8       \\ \hline
0.513       & 5.64      & 0       & 3.1       \\ \hline
0.551       & 5.53      & 0       & 3.3       \\ \hline
0.609       & 5.44      & 0       & 3.5       \\ \hline
0.655       & 5.34      & 0       & 3.7       \\ \hline
0.713       & 5.24      & 0       & 3.9       \\ \hline
0.751       & 5.16      & 0       & 4.1       \\ \hline
0.802       & 5.06      & 0       & 4.3       \\ \hline
0.856       & 5         & 0       & 4.5       \\ \hline
0.906       & 4.92      & 0       & 4.7       \\ \hline
0.955       & 4.84      & 0       & 4.9       \\ \hline
1.017       & 4.73      & 0       & 5         \\ \hline
\end{longtable}
\noindent
From the above table, we see that the maximum output voltage reached without using $\mathrm{R_c}$ is $V_o = 5 \ \mathrm{V}$ which is much less than the output voltage obtained after the breakdown voltage (close to $6V $) calculated from the previous part. Thus, it indicates that the breakdown has not been reached and explains why we are obtaining negligible current through the zener diode since breakdown has not been reached. \\[0.3cm]
\textbf{With} $\mathrm{R_c}: 2.2$ $\mathrm{k}\Omega$\\[0.3cm]
We now introduce a current limiting resistor $R_c = 2.2 \ \mathrm{k}\Omega$
\begin{longtable}{|c|c|c|c|} 
        \hline 
        $\text{R}_\text{L}$ (k$\Omega$) & {$\text{I}_\text{L}$ ($\mathrm{mA}$)} & $\text{I}_\text{A}$ (mA) & $\text{V}_\text{o}$ (V) \\ \hline \hline
        \endfirsthead
        \hline 
        {$\text{R}_\text{L}$ (k$\Omega$)} & {$\text{I}_\text{L}$ ($\mathrm{mA}$)} & {Iz ($\mathrm{mA}$)} & {$\mathrm{V_o}$ ($\mathrm{V}$)} \\ \hline \hline
        \endhead
        
        \hline
        \endfoot
        
        \hline
        \endlastfoot
        
        0.065       & 2.83      & 1.22    & 6.4     \\ \hline
        0.107       & 2.78      & 1.33    & 6.4     \\ \hline
        0.154       & 2.72      & 1.41    & 6.4     \\ \hline
        0.177       & 2.68      & 1.46    & 6.4     \\ \hline
        0.242       & 2.63      & 1.41    & 6.6     \\ \hline
        0.312       & 2.55      & 1.45    & 6.52    \\ \hline
        0.317       & 2.49      & 1.55    & 6.6     \\ \hline
        0.406       & 2.46      & 1.67    & 6.6     \\ \hline
        0.467       & 2.4       & 1.64    & 6.6     \\ \hline
        0.532       & 2.35      & 1.71    & 6.7     \\ \hline
        0.61        & 2.28      & 1.76    & 6.6     \\ \hline
        0.672       & 2.23      & 1.81    & 6.4     \\ \hline
        0.713       & 2.2       & 1.9     & 6.6     \\ \hline
        0.752       & 2.17      & 1.94    & 6.6     \\ \hline
        0.815       & 2.13      & 1.98    & 6.6     \\ \hline
        0.902       & 2.07      & 2       & 6.6     \\ \hline
        0.951       & 2.04      & 1.94    & 6.6     \\ \hline
        1.033       & 1.98      & 2.14    & 6.6     \\ \hline
        
    \end{longtable}
    
\subsection{IC 7805}
\subsubsection{Line Regulation}
For the line regulation, the resistance was kept at $\mathrm{R_c+R_L}=2.2 \ \mathrm{k}\Omega$. The input voltage was varied. The output voltage, current across IC 7805 and the current in the circuit was measured for each input voltage. The data is tabulated below:
\begin{longtable}{|c|c|c|}
        \hline
       {$\mathrm{V_{in} (V)}$} & $\mathrm{I_L (mA)}$ & $\mathrm{V_o (V)}$ \\ \hline \hline
%        \endfirsthead
    
        
    %        {$\mathrm{V_{in} (V)}$} & $\mathrm{I_L (mA)}$ & $\mathrm{V_o (V)}$ 
    %        \endhead
    
    %        \hline \hline
    %        \endfoot
        
    %        \hline
    %        \endlastfoot       
        0         & 0         & 0         \\ \hline
        0.12      & 0         & 0         \\ \hline
        0.13      & 0         & 0         \\ \hline
        0.14      & 0.01      & 0.01      \\ \hline
        0.15      & 0.76      & 0.18      \\ \hline
        1.6       & 0.85      & 0.38      \\ \hline
        1.7       & 1.04      & 0.46      \\ \hline
        1.8       & 1.17      & 0.51      \\ \hline
        2.5       & 1.76      & 0.77      \\ \hline
        3         & 1.02      & 2.23      \\ \hline
        3.4       & 1.22      & 2.66      \\ \hline
        4         & 1.47      & 3.2       \\ \hline
        4.5       & 1.67      & 3.62      \\ \hline
        5.1       & 1.92      & 4.17      \\ \hline
        5.4       & 2.03      & 4.41      \\ \hline
        5.5       & 2.12      & 4.6       \\ \hline
        5.8       & 2.21      & 4.79      \\ \hline
        5.9       & 2.25      & 4.89      \\ \hline
        6         & 2.3       & 4.99      \\ \hline
        6.5       & 2.31      & 5.0         \\ \hline
        7.1       & 2.3       & 5.0        \\ \hline
        7.5       & 2.3       & 5.0         \\ \hline
        8         & 2.3       & 5.0         \\ \hline
        9         & 2.3       & 5.0         \\ \hline
        10        & 2.3       & 5.0         \\ \hline
        11        & 2.3       & 5.0         \\ \hline
        12        & 2.3       & 5.0         \\ \hline
    \end{longtable}
\noindent
We plotted the input voltage by varying the output voltage. From the plot, we can see that the output voltage remains constant for a wide range of input voltage after $\mathrm{V_{in}=6 \ \mathrm{V}}$. Thus, from the experimental plot, we can estimate the minimum voltage of the given IC to be $6 \ \mathrm{V}$. The stable voltage obtained after the minimum applied voltage is $5 \ \mathrm{V}$ which matches with the expectation since the last digit of the IC implies so.
\subsubsection{Load Regulation}
For the load regulation, we fixed the input voltage to $V_i = 15 \ \mathrm{V}$ and varied the load resistance using a potentiometer. The output voltage, current across load was measured for each load resistance. The data is tabulated below:
\begin{longtable}{|c|c|c|}
        \hline
        $\text{R}_\text{L}$ ($\Omega$) & $\text{i}_\text{L}$ (A) & $\text{V}_0$ (V) \\ \hline \hline
        \endfirsthead
        
        \hline
        $\text{R}_\text{L}$ ($\Omega$) & $\text{i}_\text{L}$ (A) & $\text{V}_0$ (V) \\ \hline
        \endhead
        
        \hline
        \endfoot
        
        \hline
        \endlastfoot
        
        0.001      & 2.35      & 5.09 \\ \hline
        0.025      & 2.29      & 5.03 \\ \hline
        0.033      & 2.27      & 4.98 \\ \hline
        0.082      & 2.23      & 5.02 \\ \hline
        0.143      & 2.16      & 5.00 \\ \hline
        0.196      & 2.12      & 5.00 \\ \hline
        0.28       & 2.07      & 5.06 \\ \hline
        0.342      & 2.00      & 5.03 \\ \hline
        0.403      & 2.08      & 5.42 \\ \hline
        0.492      & 1.95      & 5.19 \\ \hline
        0.557      & 1.88      & 5.12 \\ \hline
        0.63       & 1.83      & 5.12 \\ \hline
        0.706      & 1.75      & 5.03 \\ \hline
        0.778      & 1.77      & 5.19 \\ \hline
        0.863      & 1.76      & 5.35 \\ \hline
        0.926      & 1.66      & 5.12 \\ \hline
        0.988      & 1.64      & 5.16 \\ \hline
        1.036      & 1.61      & 5.17 \\ \hline
        
\end{longtable}

        
\section{Conclusion}
\end{document}