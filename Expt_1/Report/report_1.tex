\documentclass{scrartcl}
\usepackage{Style_File}
\usepackage{circuitikz}
\usepackage{fancyhdr}


\setlength{\headheight}{0.75in}
\setlength{\oddsidemargin}{0in}
        \setlength{\evensidemargin}{0in}
        \setlength{\textwidth}{6.5in}
        \setlength{\headwidth}{7.3in}
        \setlength{\textheight}{8.75in}
        \rfoot{\thepage}
        \renewcommand{\headrulewidth}{0pt} % Remove the header line
        \renewcommand{\footrulewidth}{0pt}
\fancyhead[L,C]{}
\fancyhead[L]{PH3204: Experiment 1}
\fancyhead[R]{\thepage}
\fancyhead[C]{Zener Diode and IC7805}
\fancyfoot[C]{\thepage}
\fancyfoot[R,L]{}
\pagestyle{fancy}
\renewcommand{\headrulewidth}{0.4pt}


\usepackage{siunitx}
\usepackage{longtable} 
\usepackage[left = 0.5in,
right = 0.5in,
bottom = 0.9in,
top = 0.9in,
a4paper]{geometry}

\title{
        \Large\textsc{Experiment 01: }
        \huge\textbf{Study of Zener Diode and IC 7805} \\
}


\author{{\Large Sagnik Seth} -\   \texttt{22MS026}\\ ({\small Subgroup - A7}) }
\date{}


\begin{document}
\maketitle
\section{Aim}
\begin{itemize}
        \item To study Zener diode as a voltage regulator.
        \item To study IC 7805 as a voltage stabiliser.
\end{itemize}

\section{Theory}
\subsection{Zener Diode}
\begin{figure}[H]
        \centering
        \includesvg[width=0.7\textwidth]{Zener_characteristic.svg}
        \caption{Typical Zener diode characteristic curve}
\end{figure}
Zener diode is a specialised diode which works as a regular diode when forward biased but on reverse biasing, the voltage remain constant for a wide range of current after \textit{Avalanche} breakdown. Thus, Zener diode is used as a shunt voltage regulator for regulating voltage across small loads.\\[0.3cm]
Initially there is negligible current in the diode in the reverse bias condition but as breakdown voltage is reached, the current increases rapidly, however the voltage across the diode remains nearly constant.

\begin{figure}[H]
    \centering
    \begin{circuitikz}[american voltages]
        % Zener Diode with coloring
        \draw
        (5,-2) to [zzD, fill=cyan!40] (5,2)
        % Current meter and upper connections
        to [rmeter, t=mA, l_=$i_z$, fill=yellow!70] (5, 3)
        to [short, -*] (5, 4)
        to [rmeter, t=mA, l_=$i_s$, fill=yellow!70] (2, 4)
        to [R, l^=$R_s$] (1, 4)
        to [short] (0, 4)
        % Battery connection
        to [battery, l=$V_{\mathrm{in}}$] (0, -1)
        to [short] (0, -2)
        to [short] (5, -2)
        % Load branch
        (5, 4) to [short] (7, 4)
        to [R, l=$R_C$] (7, 2)
        to [rmeter, t=mA, l_=$i_L$, fill=yellow!70] (7, 1)
        to [vR, l=$R_{\mathrm{L}}$] (7, -2)
        to [short] (5, -2)
        % Probing points
        (7, 3.8) to [short, -*] (8.5, 3.8)
        (7, -1.5) to [short, -*] (8.5, -1.5);
    \end{circuitikz}
    \caption{Circuit Diagram for Load and Line Regulation Using a Zener Diode}
\end{figure}


\noindent
From the above circuit diagram, we have:
 $$I_s = I_z+I_L \implies I_s = \underbrace{\frac{V_i - V_z}{R_s}}_{I_z} + \underbrace{\frac{V_o}{R_L}}_{I_L}$$
Hence in line regulation, where $R_s$ and $R_L$ are kept constant and $V_o$ becomes constant after breakdown, from where we get $\delta I_s = \delta I_z$. Thus, we are expected to get a linear graph between $I_z$ and $I_s$ with slope 1.\\[0.3cm]
In load regulation, we keep $V_i$ constant and vary $R_L$. $I_s$ independent of load resistance, hence from above equation we get $\delta I_z = -\delta I_L$. Thus, we are expected to get a linear graph between $I_z$ and $I_L$ with slope -1.
\noindent
\subsection{Integrated Circuit based Regulator}
In Zener diode, the regulation is not perfect and output
voltage increases very slowly with increasing reverse input voltage. To resolve this, we use Integrated Circuit (IC) based voltage regulators. We will use IC 7805 for the experiment.
The 7805 IC voltage regulator has
3 pins. Pin 1 takes the input voltage and Pin 3 produces the output
voltage. The ground of both input and output are given to Pin 2.
\begin{figure}[H]
    \centering
    \begin{circuitikz}
        \draw 
        (-1,0) -- (1,0) 
        ++(1,0) node[rectangle,draw,fill=cyan!30,
          minimum width=2cm,minimum height=1.2cm,
          label={[below]north:IC7805},
          label={[right]west:In},
          label={[above]south:Grnd},
          label={[left]east:Out}]{} 
        (2,-0.6) to [short, -*] (2,-5) 
        (3,0) -- (5,0) 
        to [short, -*] (5.3,0)
        (5.3,0) to [short, -*] (6.5,0) % Junction point just above Rc
        (5.3,0) to [short] (5.3,-0.3) to [R=$R_c$] (5.3,-1.5) 
        to [rmeter, t=mA, l=$i_L$, fill=yellow!70] (5.3,-2.9)
        to [vR, l=$R_{\mathrm{L}}$] (5.3,-5)
        to [short] (-1,-5)
        to [battery, l =$V_{\mathrm{in}}$] (-1,-1)
        to [short] (-1,0)
        (5.3,-5) to [short, -*] (6.5, -5);
    \end{circuitikz}
    \caption{Circuit Diagram for load and line regulation using IC 7805}
\end{figure}


\section{Data and Calculation}
\subsection{Zener Diode:}
\subsubsection{Line Regulation }
For the line regulation, we first fixed the load resistance to $R_L = \SI{1.1}{\kohm}$ and varied the input voltage. The output voltage, current across zener diode and the current in the circuit was measured for each input voltage. The data is tabulated below:

\begin{longtable}{|l|l|l|l|}
        \hline
        $V_i$ (V) & Is (mA) & Iz (mA) & Vo (V) \\ \hline
        \endfirsthead
        \hline
        $V_i$ (V) & Is (mA) & Iz (mA) & Vo (V) \\ \hline
        \endhead
        \hline
        \endfoot
        \hline
        \endlastfoot
        0.0     & 0       & 0.00000       & 0.01    \\ \hline
        0.6    & 0       & 0.00000       & 0.4     \\ \hline
        1.0     & 0.12    & 0.00000      & 0.69    \\ \hline
        1.6    & 0.23    & 0.00000       & 1.02    \\ \hline
        2.1    & 0.35    & 0.00000      & 1.31    \\ \hline
        2.4    & 0.46    & 0.00000       & 1.52    \\ \hline
        3.0     & 0.12    & 0.00000       & 1.85    \\ \hline
        3.5    & 0.23    & 0.00000       & 2.16    \\ \hline
        3.9    & 0.58    & 0.00000      & 2.43    \\ \hline
        4.5    & 0.69    & 0.00000       & 2.77    \\ \hline
        5.0      & 0.81    & 0.00002       & 3.09    \\ \hline
        5.4    & 0.93    & 0.00008       & 3.31    \\ \hline
        6.0      & 1.05    & 0.00023       & 3.68    \\ \hline
        6.5    & 1.18    & 0.00040       & 3.92    \\ \hline
        7.0      & 1.27    & 0.00083       & 4.23    \\ \hline
        7.5    & 1.39    & 0.00161       & 4.57    \\ \hline
        8.0      & 1.51    & 0.00337       & 4.87    \\ \hline
        8.5    & 1.63    & 0.00637       & 5.16    \\ \hline
        8.9    & 1.74    & 0.01132       & 5.40    \\ \hline
        9.5    & 1.86    & 0.01666       & 5.71    \\ \hline
        10.0     & 1.98    & 0.03700       & 6.05    \\ \hline
        10.5   & 1.58    & 0.04000       & 6.35    \\ \hline
        11.0     & 2.11    & 0.10000       & 6.50    \\ \hline
        11.5   & 2.30    & 0.30000       & 6.53    \\ \hline
        11.9   & 2.50    & 0.50000       & 6.57    \\ \hline
        12.5   & 2.74    & 0.70000       & 6.55    \\ \hline
        13.0     & 3.01    & 1.00000       & 6.56    \\ \hline
        13.6   & 3.24    & 1.30000       & 6.57    \\ \hline
        13.9   & 3.41    & 1.40000       & 6.57    \\ \hline
        14.5   & 3.67    & 1.70000       & 6.57    \\ \hline
        15.0     & 3.89    & 1.90000       & 6.58    \\ \hline
        \end{longtable}
        
\noindent
\textbf{Proportionality of Zener and circuit current:}\\[0.3cm]
From the data, we plotted $Iz$ vs $Is$ and found that the graph is linear. Intially the current was very low (in range of microamperes). At around $V_{\mathrm{in}} = \SI{10.5}{\volt}$, the current started to increase rapidly (in range of milliamperes). The graph for data points after $V_{\mathrm{in}} = \SI{10.5}{\volt}$ is shown below along with linear fit curve:
\begin{figure}[H]
        \centering
        \includesvg[width=0.8\textwidth]{Zener_line_Is_vs_Iz.svg}
        \caption{Plot of $Iz$ vs $Is$ with constant load}
\end{figure}
\noindent
From the linear fit, we obtained the slope to be $m = 0.98\pm0.04$ which verifies $\delta \mathrm{Iz} = \delta \mathrm{Is}$.\\[0.3cm]
\textbf{Estimating Breakdown Voltage}\\[0.3cm]
From the above table, we plot the input vs the output voltage. 
\begin{figure}[H]
        \centering
        \includesvg[width=0.8\textwidth]{Zener_line_V0_vs_Vin.svg}
        \caption{Plot between input and output voltage with constant load}
\end{figure}
\noindent
From the plot, we can see that initially there is gradual change in the output voltage with changing input but after $V_{\mathrm{in}} = \SI{10.5}{\volt}$, the output voltage saturated and became almost constant. This indicates breakdown has occured and breakdown voltage equals to the output voltage of the circuit.
We take the average of the voltage obtained after $V_{\mathrm{in}} = \SI{10.5}{\volt}$. Thus, from the experimental plot, we estimate the breakdown voltage of the given Zener diode to be:
$$\boxed{V_\mathrm{b} = 6.55 \mathrm{V}}$$
\subsubsection{Load Regulation}
\textbf{Without $\mathrm{R_c}:$ }\\[0.3cm]
For the load regulation, we fixed the input voltage to $V_i = 15 \ \mathrm{V}$ and varied the load resistance using a potentiometer. The output voltage, current across zener diode and the load current was measured for each load resistance. The data is tabulated below:
% Include this in the preamble

\begin{longtable}{|l|l|l|l|}
\hline
$\mathrm{R_L}$ ($\mathrm{k}\Omega$) & $\mathrm{I_L}$ (mA) & Iz (mA) & $\mathrm{V_o}$ (V) \\ \hline
\endfirsthead
\hline
$\mathrm{R_L}$ ($\mathrm{k}\Omega$) & $\mathrm{I_L}$ (mA) & Iz (mA) & $\mathrm{V_o}$ (V) \\ \hline
\endhead
\hline
\endfoot
\hline
\endlastfoot
0.007       & 6.92      & 0       & 0.2       \\ \hline
0.057       & 6.76      & 0       & 0.5       \\ \hline
0.106       & 6.61      & 0       & 0.9       \\ \hline
0.153       & 6.5       & 0       & 1.2       \\ \hline
0.203       & 6.36      & 0       & 1.4       \\ \hline
0.249       & 6.25      & 0       & 1.7       \\ \hline
0.304       & 6.11      & 0       & 2.1       \\ \hline
0.354       & 6.02      & 0       & 2.3       \\ \hline
0.396       & 5.92      & 0       & 2.6       \\ \hline
0.452       & 5.78      & 0       & 2.8       \\ \hline
0.513       & 5.64      & 0       & 3.1       \\ \hline
0.551       & 5.53      & 0       & 3.3       \\ \hline
0.609       & 5.44      & 0       & 3.5       \\ \hline
0.655       & 5.34      & 0       & 3.7       \\ \hline
0.713       & 5.24      & 0       & 3.9       \\ \hline
0.751       & 5.16      & 0       & 4.1       \\ \hline
0.802       & 5.06      & 0       & 4.3       \\ \hline
0.856       & 5         & 0       & 4.5       \\ \hline
0.906       & 4.92      & 0       & 4.7       \\ \hline
0.955       & 4.84      & 0       & 4.9       \\ \hline
1.017       & 4.73      & 0       & 5         \\ \hline
\end{longtable}
\noindent
From the above table, we see that the maximum output voltage reached without using $\mathrm{R_c}$ is $V_o = 5 \ \mathrm{V}$ which is less than the output voltage obtained after reaching the breakdown voltage ( $6.55\ \mathrm{V}$) calculated from the previous part. Thus, it indicates that the breakdown has not been reached and explains why we are obtaining negligible current through the zener diode since breakdown has not been reached. \\[0.3cm]
\textbf{With $\mathrm{R_c}: 2.2 \mathrm{k}\Omega$ }\\[0.3cm]
We now introduce a current limiting resistor $R_c = 2.2 \ \mathrm{k}\Omega$

\begin{longtable}{|c|c|c|c|} 
        \hline
        $R_L$ (k$\Omega$) & {$I_L$ ($\mathrm{mA}$)} & {Iz ($\mathrm{mA}$)} & {$\mathrm{V_o}$ ($\mathrm{V}$)} \\ \hline
        \endfirsthead
        \hline 
        {$R_L$ (k$\Omega$)} & {$I_L$ ($\mathrm{mA}$)} & {Iz ($\mathrm{mA}$)} & {$\mathrm{V_o}$ ($\mathrm{V}$)} \\ \hline
        \endhead
        
        \hline
        \endfoot
        
        \hline
        \endlastfoot
        
        0.065       & 2.83      & 1.22    & 6.4     \\ \hline
        0.107       & 2.78      & 1.33    & 6.4     \\ \hline
        0.154       & 2.72      & 1.41    & 6.4     \\ \hline
        0.177       & 2.68      & 1.46    & 6.4     \\ \hline
        0.242       & 2.63      & 1.41    & 6.6     \\ \hline
        0.312       & 2.55      & 1.45    & 6.52    \\ \hline
        0.317       & 2.49      & 1.55    & 6.6     \\ \hline
        0.406       & 2.46      & 1.67    & 6.6     \\ \hline
        0.467       & 2.4       & 1.64    & 6.6     \\ \hline
        0.532       & 2.35      & 1.71    & 6.7     \\ \hline
        0.61        & 2.28      & 1.76    & 6.6     \\ \hline
        0.672       & 2.23      & 1.81    & 6.4     \\ \hline
        0.713       & 2.2       & 1.9     & 6.6     \\ \hline
        0.752       & 2.17      & 1.94    & 6.6     \\ \hline
        0.815       & 2.13      & 1.98    & 6.6     \\ \hline
        0.902       & 2.07      & 2       & 6.6     \\ \hline
        0.951       & 2.04      & 1.94    & 6.6     \\ \hline
        1.033       & 1.98      & 2.14    & 6.6     \\ \hline
        
        \end{longtable}
        \noindent
\begin{figure}[H]
        \centering
        \includesvg[width=0.7\textwidth]{Zener_load_Iz_vs_Il.svg}
        \caption{Plot of $\mathrm{I_z}$ vs $\mathrm{I_L}$ with $\mathrm{R_c} = 2.2 \mathrm{k}\Omega$}
\end{figure}
\noindent
From the above plot, we obtained the slope for the linear fit curve as: ${m = -0.98\pm0.09}$, thus verifying  $\delta\mathrm{I_z} = - \delta\mathrm{I_L}$ \\[0.3cm]
\textbf{Constancy of Output Voltage:}\\[0.3cm]
In this case, with the use of $\mathrm{R_c}$, we see that the output voltage is greater than $6.55 \mathrm{V}$, indicating that breakdown has already been reached. Thus, it remains constant for a wide range of load resistance. The plot of output voltage vs load resistance is shown below:
\begin{figure}[H]
        \centering
        \includesvg[width=0.7\textwidth]{Zener_load_Vo_vs_Rl.svg}
        \caption{Plot of $\mathrm{V_o}$ vs $\mathrm{R_L}$ with $\mathrm{R_c} = 2.2 \mathrm{k}\Omega$}
\end{figure}
\subsection{IC 7805}
\subsubsection{Line Regulation}
For the line regulation, kept the resistance $\mathrm{R_c+R_L}=2.2 \ \mathrm{k}\Omega$ and varied the input voltage. The output voltage, current across IC 7805 and the current in the circuit was measured for each input voltage. The data is tabulated below:
\begin{longtable}{|c|c|c|}
        \hline
       {$\mathrm{V_{in} (V)}$} & $\mathrm{I_L (mA)}$ & $\mathrm{V_o (V)}$ \\ \hline
        \endfirsthead
        
        \hline
        {$\mathrm{V_{in} (V)}$} & $\mathrm{I_L (mA)}$ & $\mathrm{V_o (V)}$ 
        \endhead
        
        \hline
        \endfoot
        
        \hline
        \endlastfoot
        
        0         & 0         & 0         \\ \hline
        0.12      & 0         & 0         \\ \hline
        0.13      & 0         & 0         \\ \hline
        0.14      & 0.01      & 0.01      \\ \hline
        0.15      & 0.76      & 0.18      \\ \hline
        1.6       & 0.85      & 0.38      \\ \hline
        1.7       & 1.04      & 0.46      \\ \hline
        1.8       & 1.17      & 0.51      \\ \hline
        2.5       & 1.76      & 0.77      \\ \hline
        3         & 1.02      & 2.23      \\ \hline
        3.4       & 1.22      & 2.66      \\ \hline
        4         & 1.47      & 3.2       \\ \hline
        4.5       & 1.67      & 3.62      \\ \hline
        5.1       & 1.92      & 4.17      \\ \hline
        5.4       & 2.03      & 4.41      \\ \hline
        5.5       & 2.12      & 4.6       \\ \hline
        5.8       & 2.21      & 4.79      \\ \hline
        5.9       & 2.25      & 4.89      \\ \hline
        6         & 2.3       & 4.99      \\ \hline
        6.5       & 2.31      & 5.0         \\ \hline
        7.1       & 2.3       & 5.0        \\ \hline
        7.5       & 2.3       & 5.0         \\ \hline
        8         & 2.3       & 5.0         \\ \hline
        9         & 2.3       & 5.0         \\ \hline
        10        & 2.3       & 5.0         \\ \hline
        11        & 2.3       & 5.0         \\ \hline
        12        & 2.3       & 5.0         \\ \hline
        
        \end{longtable}
\begin{figure}[H]
        \centering
        \includesvg[width=0.8\textwidth]{IC_line_Vo_vs_Vin.svg}
        \caption{Plot of $\mathrm{V_o}$ vs $\mathrm{V_i}$ for IC 7805}
\end{figure}
\noindent
We plotted the input voltage by varying the output voltage. From the plot, we can see that the output voltage remains constant for a wide range of input voltage after $\mathrm{V_{in}}=\SI{6}{\volt}$. Thus, from the experimental plot, we can estimate the minimum voltage of the given IC to be $\SI{6}{\volt}$. The stable voltage obtained after the minimum applied voltage is $5 \ \mathrm{V}$ which matches with the expectation since the last digit of the IC implies so.
\subsubsection{Load Regulation}
For the load regulation, we fixed the input voltage to $V_i = 15 \ \mathrm{V}$ and varied the load resistance using a potentiometer. The output voltage, current across load was measured for each load resistance. The data is tabulated below:
\begin{longtable}{|c|c|c|}
        \hline
        \textbf{$\mathbf{R_L}$ ($\mathbf{\Omega}$)} & \textbf{$\mathbf{i_L}$ (A)} & \textbf{$\mathbf{V_0}$ (V)} \\ \hline
        \endfirsthead
        
        \hline
        \textbf{$\mathbf{R_L}$ ($\mathbf{\Omega}$)} & \textbf{$\mathbf{i_L}$ (A)} & \textbf{$\mathbf{V_0}$ (V)} \\ \hline
        \endhead
        
        \hline
        \endfoot
        
        \hline
        \endlastfoot
        
        0.001      & 2.35      & 5.09 \\ \hline
        0.025      & 2.29      & 5.03 \\ \hline
        0.033      & 2.27      & 4.98 \\ \hline
        0.082      & 2.23      & 5.02 \\ \hline
        0.143      & 2.16      & 5.00 \\ \hline
        0.196      & 2.12      & 5.00 \\ \hline
        0.28       & 2.07      & 5.06 \\ \hline
        0.342      & 2.00      & 5.03 \\ \hline
        0.403      & 2.08      & 5.42 \\ \hline
        0.492      & 1.95      & 5.19 \\ \hline
        0.557      & 1.88      & 5.12 \\ \hline
        0.63       & 1.83      & 5.12 \\ \hline
        0.706      & 1.75      & 5.03 \\ \hline
        0.778      & 1.77      & 5.19 \\ \hline
        0.863      & 1.76      & 5.35 \\ \hline
        0.926      & 1.66      & 5.12 \\ \hline
        0.988      & 1.64      & 5.16 \\ \hline
        1.036      & 1.61      & 5.17 \\ \hline
        \end{longtable}
        \noindent
\begin{figure}[H]
        \centering
        \includesvg[width=0.8\textwidth]{IC_load_Vo_vs_Rl.svg}
        \caption{Plot of $\mathrm{V_o}$ vs $\mathrm{R_L}$ for IC 7805}
\end{figure}
\noindent
From the above plot, we can see that the output voltage remains constant for a wide range of load resistance. The output voltage remains constant at an average voltage of $\SI{5.12}{\volt}$ which matches closely with out expectation for IC7805. Thus, the IC 7805 is a good voltage stabiliser.
\section{Conclusion}
We studied the line and load regulation of Zener diode. From the line regulation, we found out the average breakdown voltage of the Zener diode to be $\SI{6.55}{\volt}$. We found out that the voltage across the diode remains almost constant for increasing input voltage once breakdown voltage is reached. \\[0.3cm]
To improve the regulation, we used IC 7805 as voltage regulator. From the line regulation, we found out the minimum voltage required for the IC to achieve a stable voltage to be $\SI{6.0}{\volt}$. The stable votage for the IC was found to be $\SI{5.0}{\volt}$ as expected.  
\end{document}