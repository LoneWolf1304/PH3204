\textbf{op-Amp as Non-Inverting Amplifier:}
\begin{figure}[H]
\begin{center}
    \begin{circuitikz}[american voltages, scale=1.2, transform shape, font=\scriptsize]
        % Op-Amp
        \node[op amp, fill=cyan!20](opamp) at (0,0) {\texttt{op-Amp}};
        
        % Input voltage source
        \draw (opamp.-) to[R, l_=$\mathrm{R_i}$, -*] ++(-2,0) node[ground] {};
        
        % Feedback resistor
        \draw (opamp.-) -- ++(0,1.2) to[R, l=$\mathrm{R_f}$] ++(2,0) -| (opamp.out);
        
        % Ground with voltage source
        \draw (opamp.+) to[short] ++(0,-1) to [sV, fill=yellow!40] ++(0,-1) node[left] {$\mathrm{V_{\text{in}}}$} node[ground] {};
        
        % Output
        \draw (opamp.out) to[short, -*] ++(1,0) node[right] {$V_{\text{out}}$};
    
        % Power supply connections
        \draw (opamp.up) to[short] ++(0,0.5) node[above]{$\mathrm{+15V}$};
        \draw (opamp.down) to[short] ++(0,-0.5) node[below]{$\mathrm{-15V}$};
    
        % Labels
        \node at (opamp.-) [above left] {$-$};
        \node at (opamp.+) [above left] {$+$};
        \node at (opamp.-) [below] {$v$};

    \end{circuitikz}
\end{center}
\caption{Circuit diagram for an non-inverting op-Amp}
\end{figure}
\noindent
Since current through op-Amp due to high resistance is almost negligible, from Kirchoff current law at the junction, we get:
$$\mathrm{\frac{-v}{R_i} = \frac{v-V_{out}}{R_f} \implies V_{out} = v\brac{1+\frac{R_f}{R_i}}} $$
From virtual ground condition, $\mathrm{v \approx V_{in}}$, hence we get an expression for the gain $\mathcal{A}$ as:
$$\mathrm{V_{out} = V_{in}\brac{1+\frac{R_f}{R_i}}}\implies \boxed{\mathcal{A} = \mathrm{\frac{V_{out}}{V_{in}}= 1+\frac{R_f}{R_i}}}$$
