\textbf{op-Amp as Adder:}
\begin{center}
\begin{figure}[H]
    \centering
    \begin{circuitikz}[american voltages, scale=1.2, transform shape, font=\scriptsize]
        % Op-Amp
        \node[op amp, fill=cyan!20](opamp) at (0,0) {\texttt{op-Amp}};
        % Input voltage sources
        \draw (opamp.-) -- ++(-1,0) coordinate (junction) {};
        \draw (junction) -- ++(0,1.5)  to[R, l_=$\mathrm{R_1}$, *-] ++(-2,0) node[left] {$\mathrm{V_1}$};
        \draw (junction) ++(0,0.0)  to[R, l_=$\mathrm{R_2}$, *-] ++(-2,0) node[left] {$\mathrm{V_2}$};
        \draw (junction)--  ++(0,-1.5)  to[R, l_=$\mathrm{R_3}$, *-] ++(-2,0) node[left] {$\mathrm{V_3}$};
        
        % Feedback resistor
        \draw (opamp.-) -- ++(0,2) to[R, l=$\mathrm{R_f}$] ++(2,0) -| (opamp.out);
        
        % Ground
        \draw (opamp.+) to[short] ++(0,-1) node[ground] {};
        
        % Output
        \draw (opamp.out) to[short, -*] ++(1,0) node[right] {$\mathrm{V_{\text{out}}}$};
    
        % Power supply connections
        \draw (opamp.up) to[short] ++(0,0.5) node[above]{$\mathrm{+15V}$};
        \draw (opamp.down) to[short] ++(0,-0.5) node[below]{$\mathrm{-15V}$};
    
        % Labels
        \node at (opamp.-) [above left] {$\mathrm{-}$};
        \node at (opamp.-) [below] {$\mathrm{V_i}$};
        \node at (opamp.+) [above left] {$\mathrm{+}$};
    \end{circuitikz}
    \caption{Circuit for op-Amp as an adder}

\end{figure}
\end{center}
\noindent
For the circuit shown above, the current through op-Amp is negligible, hence the total current through the three resistors $\mathrm{R_1}$, $\mathrm{R_2}$, $\mathrm{R_3}$ goes through $\mathrm{R_f}$. Applying Kirchoff's current law at the junction, we get: 
$$ \mathrm{\frac{V_{out} - V_i}{R_f} = \frac{V_i - V_1}{R_1} + \frac{V_i - V_2}{R_2} + \frac{V_i - V_3}{R_3} }$$
 From the virtual ground condition, $V_i \approx 0$, hence:
    $$\mathrm{ V_{out} = -  R_f\left(\frac{V_1}{R_1}  + \frac{V_2}{R_2} + \frac{V_3}{R_3} \right) }$$
    If we take $\mathrm{R_f = R_1 = R_2 = R_3}$, then:
    $$ \boxed{\mathrm{V_{out} = - (V_1 + V_2 + V_3) }}$$
    Hence, the circuit acts as an adder, that is, it adds the individual input voltages.\\[0.3cm]