\begin{figure}[H]
    \centering
    \begin{circuitikz}[american voltages, scale=1.2, transform shape, font=\scriptsize]
        % Op-Amp
        \node[op amp, fill=cyan!20](opamp) at (0,0) {\texttt{op-Amp}};
        
        % Input V1 (to the inverting terminal)
        \draw (opamp.-) to[R, l=$\mathrm{R_1}$, *-] ++(-5,0) to[sV, l=$\mathrm{V_1}$, fill=yellow!40] ++(0,-2) node[ground] {};
        
        % Input V2 (to the non-inverting terminal)
        \draw (opamp.+) to[R, l=$\mathrm{R_2}$, *-] ++(-3,0) to[sV, l=$\mathrm{V_2}$, fill=yellow!40] ++(0,-2) node[ground] {};
        
        % Feedback resistor
        \draw (opamp.-) -- ++(0,1.5) to[R, l=$\mathrm{R_3}$] ++(2,0) -| (opamp.out);
        
        % Ground for the non-inverting terminal
        \draw (opamp.+) -- ++(0,-1.5) to[R, l=$\mathrm{R_4}$] ++(0,-1.5) node[ground] {};
        
        % Output
        \draw (opamp.out) to[short, -*] ++(1,0) node[right] {$\mathrm{V_{\text{out}}}$};
    
        % Power supply connections
        \draw (opamp.up) to[short] ++(0,0.5) node[above]{$\mathrm{+15V}$};
        \draw (opamp.down) to[short] ++(0,-0.5) node[below]{$\mathrm{-15V}$};
    
        % Labels
        \node at (opamp.-) [above left] {$\mathrm{-}$};
        \node at (opamp.+) [above left] {$\mathrm{+}$};
    \end{circuitikz}
    \caption{Circuit diagram for an op-Amp as a subtractor}
\end{figure}