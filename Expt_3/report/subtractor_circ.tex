\linebreak
\textbf{op-Amp as Subtractor:}
\begin{figure}[H]
    \centering
    \begin{circuitikz}[american voltages, scale=1.2, transform shape, font=\scriptsize]
        % Op-Amp
        \node[op amp, fill=cyan!20](opamp) at (0,0) {\texttt{op-Amp}};
        
        % Input V1 (to the inverting terminal)
        \draw (opamp.-) to[R, l=$\mathrm{R_1}$, *-] ++(-5,0) to[sV, l=$\mathrm{V_1}$, fill=yellow!40] ++(0,-2) node[ground] {};
        
        % Input V2 (to the non-inverting terminal)
        \draw (opamp.+) to[R, l=$\mathrm{R_2}$, *-] ++(-3,0) to[sV, l=$\mathrm{V_2}$, fill=yellow!40] ++(0,-2) node[ground] {};
        
        % Feedback resistor
        \draw (opamp.-) -- ++(0,1.5) to[R, l=$\mathrm{R_3}$] ++(2,0) -| (opamp.out);
        
        % Ground for the non-inverting terminal
        \draw (opamp.+) -- ++(0,-1.5) to[R, l=$\mathrm{R_4}$] ++(0,-1.5) node[ground] {};
        
        % Output
        \draw (opamp.out) to[short, -*] ++(1,0) node[right] {$\mathrm{V_{\text{out}}}$};
    
        % Power supply connections
        \draw (opamp.up) to[short] ++(0,0.5) node[above]{$\mathrm{+15V}$};
        \draw (opamp.down) to[short] ++(0,-0.5) node[below]{$\mathrm{-15V}$};
    
        % Labels
        \node at (opamp.-) [above left] {$\mathrm{v_1}$};
        \node at (opamp.+) [above left] {$\mathrm{v_2}$};
    \end{circuitikz}
    \caption{Circuit diagram for op-Amp as a subtractor}
\end{figure}
\noindent
For the circuit shown above, the current through op-Amp is negligible. Thus, using Kirchoff's current law at the junctions, we get:
\begin{align*}
    \mathrm{\frac{V_{1} - v_1}{R_1}} &= \mathrm{\frac{v_1 - V_{\text{out}}}{R_3} = \frac{V_1 - V_{\text{out}}}{R_1+R_3}\implies  v_1 = \frac{R_3}{R_1+R_3}(V_1 - V_{\text{out}})+V_{\text{out}}}\\[0.4cm]
    \mathrm{\frac{V_{2} - v_2}{R_2}} &= \mathrm{\frac{v_2-0}{R_4} = \frac{V_2}{R_2+R_4}\implies  v_2 = \frac{R_4}{R_2+R_4}V_2}\\
\end{align*}
From the virtual ground condition, $\mathrm{v_1 \approx v_2}$, hence:
\begin{align*}
    \mathrm{\frac{R_3}{R_1+R_3}V_1+\frac{R_1}{R_1+R_3}V_{\text{out}} = \frac{R_4}{R_2+R_4}V_2\implies V_{out} = \frac{R_4(R_1+R_3)}{(R_4+R_2)R_1}V_1 - \frac{R_3}{R_1}V_2}
\end{align*}
If we take $\mathrm{R_1 = R_2 = R_3 = R_4}$, then:
\begin{align*}
    \boxed{\mathrm{V_{\text{out}} = -(V_2 - V_1)}}
\end{align*}
Hence, the circuit acts as a subtractor, that is, it subtracts the individual input voltages.